\subsection{Model Problems from Verbal Instruction}

% add outline page with current section highlighted.
\begin{frame}{Outline}{ $ \null $ }
	%\tableofcontents[currentsection]
	\tableofcontents[currentsection,currentsubsection]
\end{frame}

%\begin{frame}{The Spatial Hierarchy}{Related Work}

%\begin{figure}
%	\centering
%	\includegraphics[width=.45\linewidth]{figure/semantic_hierarchy}
%	\caption{ \tiny{ {\it Kuipers et al. } ``The spatial semantic hierarchy.'' Artificial intelligence 2000 } }
%\end{figure}

%\end{frame}

\begin{frame}{Parse the Instruction}{Related Work}

\begin{columns}
\column{0.48\textwidth}
	\begin{block}{Spatial Description Clause}
	\begin{minipage}[t][5.7cm][t]{\textwidth}
		\begin{figure}
			\centering
			\includegraphics[width=\linewidth]{figure/spatial_description_clause}
			\caption{ \tiny{ {\it Kollar et al.} ``Toward understanding natural language directions.'' 5th ACM/IEEE International Conference on Human-Robot Interaction (HRI) 2010.} }
		\end{figure}
	\end{minipage}
	\end{block}

\column{0.48\textwidth}
	\begin{block}{Tactical Behavior Specification}
    \begin{minipage}[t][5.7cm][t]{\textwidth}
		\begin{figure}
			\centering
			\includegraphics[width=\linewidth]{figure/tactical_behavior_specification}
			\caption{ \tiny{ {\it Boularias et al. } ``Grounding spatial relations for outdoor robot navigation.'' 2015 IEEE International Conference on Robotics and Automation (ICRA) 2015. } }
		\end{figure}
	\end{minipage}
	\end{block}
\end{columns}

\end{frame}

%\begin{frame}{Inferring Information from Language Instructions}{Related Work}
%\begin{figure}
%	\centering
%    \includegraphics[width=.7\linewidth]{figure/inferring_information_from_language}
%	\caption{ \tiny{ {\it Duvallet et al. } ``Inferring maps and behaviors from natural language instructions.'' International Symposium on Experimental Robotics (ISER)  2014. } }
%\end{figure}
%\end{frame}

\begin{frame}{Language Understanding}{Related Work}

{\bf G3 Model (Generalized Grounding Graph)\footnotemark }

\begin{figure}
	\centering
	\includegraphics[width=.6\linewidth]{figure/G3}
	\caption{ \tiny{ {\it Howard et al.} ``A natural language planner interface for mobile manipulators.'' 2014 IEEE International Conference on Robotics and Automation (ICRA) 2014. } }
\end{figure}
\begin{equation}
\nonumber
\arg \max_{\Gamma} p( \Phi = True \mid \Lambda , \Gamma ) 
\end{equation}
\begin{itemize}
\item $ \Lambda $ - command, $ \Gamma $ - groundings
\item $ \Phi $ enables domain-independent learning and inference
\end{itemize}

\footnotetext[5]{\tiny {\it Tellex et al.} ``Understanding Natural Language Commands for Robotic Navigation and Mobile Manipulation.'' AAAI 2011.}

\end{frame}

\begin{frame}{Language Understanding}{Related Work}
	
	{\bf G3 Model (Generalized Grounding Graph) }
	
	\begin{figure}
		\centering
		\includegraphics[width=.6\linewidth]{figure/G3}
		\caption{ \tiny{ {\it Howard et al.} ``A natural language planner interface for mobile manipulators.'' 2014 IEEE International Conference on Robotics and Automation (ICRA) 2014. } }
	\end{figure}
	
	\begin{equation}
	\nonumber
	\begin{aligned}
	& \arg \max_{\Gamma} p( \Phi = True \mid \Lambda , \Gamma )  \\
	= & \arg \max_{\gamma_i \cdots \gamma_n} \prod_i p( \phi = True \mid \gamma_i , \lambda_i , \Gamma_{c_i} )
	\end{aligned}
	\end{equation}
	
	{\bf Shortcoming} - number of groundings grows exponentially
	
	
\end{frame}

\begin{frame}{Language Understanding}{Related Work}

{\bf DCG Model (Distributed Correspondence Graph)\footnotemark }

\begin{columns}
\column{.05\textwidth}
\column{.35\textwidth}
\begin{figure}
	\centering
	\includegraphics[width=\textwidth]{figure/DCG}
	\caption{ \tiny{ {\it Howard et al.} ``A natural language planner interface for mobile manipulators.'' 2014 IEEE International Conference on Robotics and Automation (ICRA) 2014. } }
\end{figure}
\column{.6\textwidth}
\begin{itemize}
	\item all phrases are conditionally independent
	\item all groundings are composed of conditionally independent elements
	\item 
	\begin{equation}
	\nonumber
	\arg \max_{ \phi_{ij} } \prod_i p( \phi_{ij} \mid \gamma_{ij} , \lambda_i,  \Gamma_{c_{ij}} )
	\end{equation}
\end{itemize}
\end{columns}

\footnotetext[6]{\tiny{ {\it Howard et al.} ``A natural language planner interface for mobile manipulators.'' 2014 IEEE International Conference on Robotics and Automation (ICRA) 2014. }}

\end{frame}

\begin{frame}{Language Understanding}{Related Work}

{\bf Hybrid G3-DCG Model }

\begin{figure}
	\centering
	\includegraphics[width=.35\linewidth]{figure/hybrid_G3_DCG}
	\caption{ \tiny{ {\it Howard et al.} ``A natural language planner interface for mobile manipulators.'' 2014 IEEE International Conference on Robotics and Automation (ICRA) 2014. } }
\end{figure}

	\begin{equation}
	\nonumber
	\arg \max_{ \gamma_i , \phi_{ij} } \prod p( \phi_i \mid \gamma_i , \lambda_i , \Gamma_{c_i}  ) \prod p( \phi_{ij} \mid \gamma_{ij} , \lambda_i,  \Gamma_{c_{ij}} ) 
	\end{equation}

\end{frame}