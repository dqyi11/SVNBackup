\section{Related Work}

There exists a huge amount of variations on particle swarm optimization algorithms in different types of problems. 

\cite{4223164} defines a standard for the PSO, which includes a ring topology, a constricted update rule and etc. 
The constricted update rule is written as

\begin{subequations}
\label{eq:pso_alg}
\begin{equation}
\label{eq:up_vel}
\begin{aligned}
\vec{v}_{i}(k+1) & = \chi [ \vec{v}_{i}(k) 
 + \phi^{P} \vec{u}^{P}_{i}(k) \otimes (\vec{x}^{P}_{i}(k) - \vec{x}_{i}(k)) \\
& + \phi^{G} \vec{u}^{G}_{i}(k) \otimes (\vec{x}^{G}_{i}(k) - \vec{x}_{i}(k)) ],
\end{aligned}
\end{equation}
\begin{equation}
\label{eq:up_pos}
\vec{x}_{i}(k+1) = \vec{x}_{i}(k) + \vec{v}_{i}(k+1).
\end{equation}
\end{subequations}

As there exists a social interaction between the swarm particles and the random factor in the constricted update rule, the dynamic of the problem usually shows some non-deterministic property, like many other evolutionary algorithms.
Understanding the system dynamic helps 
A class of analysis focuses on the convergence of the PSO algorithms. 

\cite{Clerc06stagnationanalysis} proposed the ``stagnation phenomenon'', in which the exploration process is not able to find a new improvement.

The analysis methods can be categorized into two approaches.

One approach is ignoring the stochastic factors, in which the random terms are replaced with constant terms.
\cite{985692} proposed how a particle will converge when it is in a stagnation.
Based on the convergence analysis, \cite{Trelea2003317} proposed how to use that as the guidance on parameter selection.


The other approach focuses on the stochastic analysis.
Naturally, after applying the mean to the variables, the stochastic terms are converted into constant terms again. 

Using a continuous time model to approximate the discrete time dynamic of the PSO, \cite{5675669} imported a system dynamic from the continuous model to simulate the particle trajectory. 

From a discrete-time model, 
\cite{Jiang20078} 
\cite{jiang2007particle}
modeled the dynamics of the mean and the variance of a particle in the stagnation.
Using the characteristic equation, the convergence analysis is obtained.
In a similar way, 
\cite{5175367}
\cite{Poli:2007:EAS:1276958.1276977}
\cite{Poli:2008:DSS:1384929.1384944}
modeled the different order moment into discrete-time system.
Using the discrete-time system model of different moments, the equilibrium can be found.
The stability requirements are also obtained from the norm of the root values of the characteristic equation are all less than 1.


Viewing the system as random search process, 
\cite{vandenBergh:2010:CPP:2010420.2010421} analyze the the probability of the convergence along the time.

By building a linear system model, \cite{4424687} viewed the PSO algorithm as a closed loop system .
Moreover, \cite{Chakraborty20111411} applied this modeling method into a multi-objective problem using the PSO algorithm. 

Usually the analyses are taken under the assumption that the system reaches the stagnation, which often means that the global best and the personal best are not changing. 
It means that the optimal search arrives at a local optimum or a global optimum.
By looking at the step, \cite{Schmitt:2013:PSO:2463372.2463563} shows how a particle might reach to at least a local optimum.

Converting the stochastic terms in equation \eqref{eq:pso_alg} into constant terms does not consider the effect from the stochastic terms, which is an important feature of the PSO algorithm.
In this paper, we analyze the convergence of the PSO without converting the stochastic terms into constant terms.

Stagnation analysis is limited only after the swarm reaches stagnation.
Whether and how the swarm can reaches a stagnation is still unknown.
In this paper, we extend the convergence analysis from only within stagnation assumption. 