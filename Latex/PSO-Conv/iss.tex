\section{Input-to-State Stability}
\label{sec:iss}

In order to help the proof, we import several comparison functions.

\begin{mydef}[$ K $-function]
\label{def:k_func}
A continuous function $ \gamma : \mathbb{R}_{geq 0} \rightarrow \mathbb{R}_{geq 0} $ is a $ K $-function if it is continuous, strictly increasing and $ \gamma (0) = 0 $. \\
It is a $ K_{\infty} $-function if it is a $ K $-function and $ \gamma (s) \rightarrow \infty $ as $ s \rightarrow \infty $.
\end{mydef}

\begin{mydef}[KL-function]
\label{def:kl_func}
A continuous function $ \beta : \mathbb{R}_{geq 0} \times \mathbb{R}_{geq 0} \rightarrow \mathbb{R}_{geq 0} $ is a $ KL $-function if for each fixed $ t \geq 0 $, the function $ \beta ( \dot , t ) $ is a $ K $-function, and for each fixed $ s \geq 0 $, the function $ \beta ( s, \dot ) $ is decreasing and $ \beta (s, t) \rightarrow 0 $ as $ t \rightarrow \infty $.
\end{mydef}

\begin{mydef}[Positive definite function]
\label{def:pd_func}
A function $ \gamma $ is \emph{positive definite} if $ \gamma (s) > 0, \forall s > 0 $ and $ r(0)= 0 $.
\end{mydef}

\begin{mydef}[Input-to-state stable]\cite{Jiang2001857}
\label{def:iss}
For a discrete-time system
\begin{equation}
\label{eq:dis_nonlinear}
x(k+1) = f( x(k) , u(k) ),
\end{equation}
with $ f(0,0) = 0 $
\footnote{It means that $ x = 0 $ is an equilibrium of the 0-input system.}, the system is \emph{(globally) input-to-state stable} if there exist a $ KL $-function $ \beta  $ and a $ K $-function $ \gamma $ such that, for each input $ u \in l^{m}_{\infty} $ and each $ \xi \in \mathbb{R}^{n} $, it holds that
\begin{equation}
\label{eq:def_iss}
| x(k, \xi, u) | \leq \beta (| \xi |, k) + \gamma (|| u ||)
\end{equation}
for every $ k \in \mathbb{Z}_{+} $.
\end{mydef}

The input-to-state stability indicates that the state of the system is bounded in a range determined by the bound of the input.
The $ \beta () $ term in equation \eqref{eq:def_iss} defines an initial bound with a decaying feature.
The $ \gamma () $ term in equation \eqref{eq:def_iss} defines a bound determined by the input.
It means that the influence $ \beta () $ term will gradually decrease to zero along time and the state will be bounded by a range determined by the bound of the input after that.

Recall the ``explore and exploit'' property in the PSO algorithm, a bound of the particles' states indicates the balance between the explore and the exploit.
We do not expect to see that the states converge while the personal best and global best have converged.
In the way we decomposed the PSO algorithm, input-to-state stability provides what we expect to see in the update rule sub-system.
Different algorithm provides different optimization strategy sub-systems. 
When it is expected to maintain a certain capability of the explore, the states of the particles could be managed in a boundary.
When it is expected to finish exploring and reach a stagnation, the states of the particles could converge.

\begin{mydef}[ISS-Lyapunov function]\cite{Jiang2001857}
\label{eq:iss_func}
A continuous function $ V : \mathbb{R}^{n} \rightarrow \mathbb{R}_{\geq 0} $ is an \emph{ISS-Lyapunov function} for system \eqref{eq:dis_nonlinear} if the followings hold:
\begin{enumerate}
\item there exist $ K $-functions $ \alpha_{1} $, $ \alpha_{2} $ such that 
\begin{equation}
\label{eq:iss_lyapunov1}
\alpha_{1} ( | \xi | ) \leq V( \xi ) \leq \alpha_{2}  ( | \xi | ), \forall \xi \in \mathbb{R}^{n},
\end{equation}
\item there exist a $ K_{\infty} $-function $ \alpha_{3} $ and a $ K $-function $ \sigma $, such that
\begin{equation}
\label{eq:iss_lyapunov2}
V( f( \xi, \mu ) ) - V( \xi ) \leq - \alpha_{3} ( | \xi | ) + \sigma ( | \mu | ),
\forall \xi \in \mathbb{R}^{n}, \mu \in \mathbb{R}^{m}.
\end{equation}
\end{enumerate}
\end{mydef}

\begin{mylem}\cite{Jiang2001857}
\label{lem:iss_stable}
If system \eqref{eq:dis_nonlinear} admits a continuous ISS-Lyapunov function, then it is input-to-state stable.
\begin{proof}
The proof is given in Appendix \ref{sec:app:proof_lemma}.
\end{proof}
\end{mylem}


\begin{mythm}
The system \eqref{eq:pso_up_linalg} is input-to-state stable, if there exists a symmetric positive definite matrix $ P $ and a symmetric positive definite matrix $ Q $ that has $ A^{T} P A - P = - Q $ , in which 
\begin{equation}
A = 
\begin{bmatrix}
\chi & - \lambda \\
\chi & 1 - \lambda
\end{bmatrix}
\end{equation}
and $ \lambda \in [0, \chi ( \phi^{G} + \phi^{P} ) ] $.

\begin{proof}


\end{proof}
\end{mythm}

\begin{mycoro}
\label{coro:state_bound}
\begin{equation}
\label{eq:state_bound}
| x(k) | \leq \max ( x(0) , \alpha_{1}^{-1} \circ \alpha_{2} \circ \alpha_{3}^{-1} \circ \sigma ( || u || ) )
\end{equation}
\begin{proof}
Let $ \rho = Id $.
With $ x( k_{0} ) \in D $ and $ b = \alpha_{4}^{-1} \circ \sigma ( || u || ) $,
we have $ j_{0} = \min \{ k \in \mathbb{Z}_{+} \mid x(k) \in D \} $.
When $ k \geq j_{0} $, $ V( x(k) ) \leq \alpha_{4}^{-1} \circ \sigma ( || u ||  ) $.
When $ k < j_{0} $, $ V( x(k) ) \leq V( x(0) ) $.
Thus, we can have equation \eqref{coro:state_bound}.
\end{proof}
\end{mycoro}