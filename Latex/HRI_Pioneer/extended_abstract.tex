\documentclass{acm_proc_article-sp}

\usepackage{comment}

\begin{document}

\title{Human Interactive Multi-objective Path Planning}
\subtitle{[Extended Abstract]
%\titlenote{A full version of this paper is available as
%\textit{Author's Guide to Preparing ACM SIG Proceedings Using
%\LaTeX$2_\epsilon$\ and BibTeX} at
%\texttt{www.acm.org/eaddress.htm}}
}

\numberofauthors{8} 
\author{
% 1st. author
\alignauthor
Daqing Yi \\%\titlenote{Dr.~Trovato insisted his name be first.}\\
       \affaddr{Brigham Young University}\\
       %\affaddr{1932 Wallamaloo Lane}\\
       \affaddr{Provo UT, United States}\\
       \email{daqing.yi@byu.edu}
}

%\additionalauthors{Additional authors: John Smith (The Th{\o}rv{\"a}ld Group,
%email: {\texttt{jsmith@affiliation.org}}) and Julius P.~Kumquat
%(The Kumquat Consortium, email: {\texttt{jpkumquat@consortium.net}}).}
%\date{30 July 1999}
\date{}

\maketitle
\begin{abstract}
\end{abstract}

\begin{comment}
Extended abstract — an anonymized, two-page description of the applicant's past, current, or proposed work (templates available) covering:
(1)the key research questions/motivation of the applicant's work,
(2)background and related work that informs the research,
(3)the research approach and methodology, and
(4)results to date (if any) and a description of remaining work.
\end{comment}

\section{Introduction}
% the key research questions/motivation of the applicant's work
There usually exists 

\section{Related work}
% background and related work that informs the research


\section{Methodology}
% the research approach and methodology

\section{Results}
% results to data (if any) and a description of remaining work

\bibliographystyle{abbrv}
\bibliography{reference}

\end{document}