\section{Introduction}
\label{sec:intro}

This project is based on ``Distributed cooperative attitude synchronization and tracking for multiple rigid bodies'' by Dr. Ren \cite{5229134}, in which discusses the consensus seeking of a system of agents with non-linear model.
Two approaches have been proposed for ``cooperative attitude synchronization'' (position control) and ``reference attitude tracking'' (trajectory tracking) respectively.
Two approaches are stated independently and look different.
One objective of this project is seeking the consensus between two approaches, which can be used as a pattern on analyzing other synchronization systems and designing control laws for them.

The structure of this report is as following.
Section \ref{sec:prob_stat} states the system model, which includes the rigid body dynamics of each agent and the interaction topology.
Section \ref{sec:coop_att_syn} presents the control law proposed for the cooperative attitude synchronization and the proof strategy.
%I try to figure out the roles of the terms in the control law so that I can simplify the control law with ignorance on the transient performance and prove the stability.
By analyzing the roles of the terms in the control law, I simplify the control law to find the essential component that guarantees the stability.
It means that the stability can still be preserved and proved without some terms for transient performance.
Section \ref{sec:ref_att_trk} illustrates the control law proposed for the reference attitude tracking and the proof strategy.
I use a same proof strategy to prove the stability of the control law on the cooperative attitude synchronization, which shows the consistency between two approaches.
It means that the ideas of improvement can be applied to both control approaches.
Simulations are also taken on both approaches for analysis purpose.
Section \ref{sec:io_stable} proposes an interconnection perspective for analyzing a class of distributed system control problems as a pattern of analyzing a consensus-seeking problem.
A hypothesis on proving the stability is proposed, in which the proof can be achieved by analyzing the stability-relevant properties of the sub-components.
Section \ref{sec:summary} discusses some possible future work from Dr. Ren's paper.