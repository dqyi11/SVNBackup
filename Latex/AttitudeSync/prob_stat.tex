\section{Problem Statement}
\label{sec:prob_stat}

\subsection{Rigid body attitude dynamics}
\label{sec:att_dyn}

In a multi-agent system, each agent adopts a rigid body attitude dynamics.
For an agent $ i $, let $ \phi_{i}  \in \mathbb{R}^{3} $ be its angle, $ \omega_{i} \in \mathbb{R}^{3} $ be its angular velocity, 
$ J_{i} \in \mathbb{R}^{3} \times \mathbb{R}^{3} $ be its inertia and $ \tau_{i} \in \mathbb{R}^{3} $ be its control torque.
Modified Rodriguez parameters(MPR) is introduced to represent attitude, which is $ \sigma_{i} = a_{i} \tan( \frac{ \phi_{i} }{4} ) $.
The attitude dynamics of agent $ i $ is transformed into 
\begin{subequations}
\begin{equation}
 \dot{\sigma}_{i} = F( \sigma_{i} ) \omega_{i},
\end{equation}
\begin{equation} 
  J_{i} \dot{\omega}_{i} = - \omega_{i} \times J_{i} \omega_{i} + \tau_{i},
\end{equation}  
\end{subequations}  
where $ F( \sigma_{i} ) = \frac{1}{2} \left[  \frac{ 1 - \sigma^{T}_{i} \sigma_{i} }{2} I_{3} + \sigma_{i}^{ \times } + \sigma_{i} \sigma^{T}_{i} \right] $
\footnote{ For a vector $ v = [ v_{1}, v_{2}, v_{3} ]^{T} $, $ v^{\times} = \begin{bmatrix} 0 & - v_{3} & v_{2} \\ v_{3} & 0 & - v_{1} \\ - v_{2} & v_{1} & 0 \end{bmatrix} $. It means that $ v \times w = v^{\times} w $. }.

It can be written in another form 
\begin{equation}
H^{*}_{i} ( \sigma_{i} ) \ddot{\sigma}_{i} + C^{*}_{i} ( \sigma_{i} , \dot{ \sigma }_{i} ) \dot{ \sigma }_{i} = F^{-T} ( \sigma_{i} ) \tau_{i} ,
\end{equation}
where $ H^{*}_{i} ( \sigma_{i} ) = F^{-T} ( \sigma_{i} ) J_{i} F^{-1} ( \sigma_{i} ) $ and
$ C^{*}_{i} ( \sigma_{i} , \dot{ \sigma }_{i} ) = - F^{-T} ( \sigma_{i} ) J_{i} F^{-1} ( \sigma_{i} ) \dot{F} ( \sigma_{i} ) F^{-1} ( \sigma_{i} ) - F^{-T} ( \sigma_{i} ) ( J_{i} F^{-1} ( \sigma_{i} ) \dot{ \sigma_{i} } )^{ \times } F^{-1} ( \sigma_{i} ) $.
Several properties of these equations have been used to facilitate the control design.
\begin{itemize}
\item $ H^{*}_{i} ( \sigma_{i} ) $ is a symmetric positive-definite matrix;
\item $ \dot{H}^{*}_{i} ( \sigma_{i} ) - 2 C^{*}_{i} ( \sigma_{i} , \dot{ \sigma }_{i} ) $ is a skew-symmetric matrix;
\item $ \tau_{i} = F^{T}( \sigma_{i} ) \tau^{'}_{i} $ makes $ H^{*}_{i} ( \sigma_{i} ) \ddot{\sigma}_{i} + C^{*}_{i} ( \sigma_{i} , \dot{ \sigma }_{i} ) \dot{ \sigma }_{i} = \tau^{'}_{i} $.
\end{itemize}

\subsection{The topology of a multi-agent system}
\label{sec:sys_topo}

The synchronization process relies on the interaction between agents in the system.
Dr. Ren's paper \cite{5229134} uses a virtual leader, the dynamics of which is not influenced by other agents and is used as the system control objective.
It is assumed that usually only a small subset of the agents can interact/communicate with the virtual leader due to the communication cost.
The synchronization process becomes how the control objectives can be propagated from the virtual leader to all other agents.  

There are two types of interaction topology, the directed graph and the undirected graph, which have symmetric and asymmetric adjacency matrices respectively.
The symmetry property is important in stability proof used in solving the cooperative attitude synchronization problem.
A symmetric graph Laplacian matrix $ M $ can have $ M^{T} M^{-1} = I $.
Moreover, it is assumed that the edges are weighted and there is no self-edge.
In order to guarantee the information propagation and facilitate the problem analysis, it is assumed that there is a path from a virtual leader to any other vertex.
\begin{itemize}
\item \textbf{Undirected graph}: The graph is connected, including the vertex of the virtual leader [Theorem 3.1 \cite{5229134}];
\item \textbf{Directed graph}:  There exists a directed spanning tree, in which the vertex of the virtual leader is the root [Theorem 4.1 \cite{5229134}].
\end{itemize}

There are two forms of information propagation on the interaction topology.
The \emph{direct} propagation is by relaying the reference attitude from the virtual leader.
The \emph{indirect} propagation is by synchronizing with neighboring agents. 
%The neighborhood is determined by the edges of the interaction topology.
The advantages of an \emph{indirect} propagation come from the independence of a routing system.
\begin{itemize}
\item \textbf{Scalability}: The system can be easily expanded into a large agent number;
\item \textbf{Flexibility}: The system can adapt to a time-varying interaction topology;
\item \textbf{Robustness}: The transient performance can be guaranteed due to the coupling effects from the neighbors.
\end{itemize}

