\section{Discussion}
\label{sec:summary}

There are some potential future work on Dr. Ren's paper \cite{5229134}.
\begin{itemize}
\item $ - F^{T}(\sigma_{i}) $ is used in designing the control law to eliminate the $ F^{-1}(\sigma_{i}) $ in the rigid body attitude dynamics.
It means that the performance of the control law relies on the correctness of the model.
One possible improvement might be importing an adaptive control or a sliding mode control (probably not good for aerial vehicle control though). 
However, this is not the focus of this project and is not discussed due to the size limitation.
\item The attitude synchronization depends on an undirected graph topology, because the symmetry of the graph Laplacian is needed in proving the stability.
It can be extended to support a directed graph topology.
\item As all the agents are aerial vehicles, the role of a damping term in the control is sometimes very important.
In simulating the reference tracking, because the reference trajectory is not very smooth, I do not import a damping term to tune the performance.
If needed, the idea of a damping term in attitude synchronization can be imported to reference tracking as well.
\end{itemize}

How to analyze a consensus seeking problem and design a control law are summarized as
\begin{itemize}
\item combining the states of all the agents to get a system state from a parallel interconnection perspective;
\item decomposing the system into a coordinate scheme and a consensus scheme from a serial feedback interconnection perspective.
\end{itemize}
%These ideas can be applied to many distributed optimization problems and distributed control problems.
If the hypothesis \ref{hyp:one} proposed in Section \ref{sec:io_stable} can be theoretically proved, it can be applied to many distributed algorithms that have consensus-seeking spirits.