%sample file for Modelica 2011 Abstract page

\documentclass[12pt,a4paper]{article}
\usepackage{graphicx}
% uncomment according to your operating system:
% ------------------------------------------------
\usepackage[latin1]{inputenc}    %% european characters can be used (Windows, old Linux)
%\usepackage[utf8]{inputenc}     %% european characters can be used (Linux)
%\usepackage[applemac]{inputenc} %% european characters can be used (Mac OS)
% ------------------------------------------------
\usepackage[T1]{fontenc}   %% get hyphenation and accented letters right
\usepackage{mathptmx}      %% use fitting times fonts also in formulas
% do not change these lines:
\pagestyle{empty}                %% no page numbers!
\usepackage[left=35mm, right=35mm, top=15mm, bottom=20mm, noheadfoot]{geometry}
%% please don't change geometry settings!


% begin the document
\begin{document}
\thispagestyle{empty}

\title{\textbf{Supporting Task-Oriented Collaboration in Human-Robot Teams Using Semantic-based Path Planning}}
\author{Daqing Yi \quad Michael A. Goodrich\\
Brigham Young University\\
%Provo, UT 84602\\
%daqing.yi@byu.edu \quad mike@cs.byu.edu
}
\date{} % <--- leave date empty
\maketitle\thispagestyle{empty} %% <-- you need this for the first page


%(1) How the story is like? (Why it is important or meaningful?) \\
%(2) How to analyze a high-level command, and how to abstract it into a quantitative model? 
%(Using an example to describe here? As you said, "Screen the back door") \\
%(3) How to model the problem into a multi-objective optimization problem? \\
%(4) Provide performance metric for algorithms \\

As robot sensing, perception and decision-making improves, the human's role in human-robot interaction progressively shifts from teleoperator to supervisor to teammate. This shift toward human-robot teaming means that concepts from human-human teaming become more important, particularly the role of shared mental models to facilitate mutual understanding and ground communication. Of particular importance is how shared mental models enable a more collaborative approach to problem solving, facilitated by communications that operate at a higher, more tactical or strategic level of abstraction.  

When human-robot teaming requires task-oriented collaboration in the real world, it is often necessary for the robot to translate a human's verbal command into a sequence and schedule of robot motions. \emph{Task grammars} and \emph{semantic labels} of objects in the world enable this translation and therefore serve as the basis for how a robot builds a mental model to understand the human's commands. For example, if a human tells a robot to ``carefully screen the back door'', this command defines the task as a ``screen'' task, defines ``back door'' as constraining the task to  specific work region, and defines ``carefully'' as objective functions that dictates how the task is done (e.g., avoid collisions and move smoothly). We assume a task grammar that specifies a task, one or more constraints, and one or more adverbs that specify how the task should be performed or how the constraints should be managed.

More generally, a verbal command from a human contains multiple objectives and constraints, which means that the robot's path-planning problem is a multi-objective optimization problem. Following related work on blending metric-based and topological approaches to path planning, we propose a two-layer approach by which a robot interprets the human's command and translates this into a sequence of motions. A waypoint layer generates a sequence of waypoints that satisfy the task objectives and constraints. Given this coarse level of planning, the motion planning layer generates the trajectories connecting two neighboring waypoints in a way that satisfies the adverbial modifiers to the task or constraints.

Multiple objectives inherently exist in both layers of planning because adverbial modifiers from the human do not ``cleanly'' map to unique metric-based performance objectives. We present an efficient framework of optimized path-planners that can be flexible and adaptive to new forms of objective definitions from new scenarios and new information sources. Furthermore, we claim that the solutions generated by these planners are \emph{Pareto optimal}, which means no objective can be improved without degrading other objectives. This property is important because it means that information communicated from the planners to the human can represent the inherent tradeoffs in satisfying the adverbial modifier and allow the human to refine their intent by selecting among these tradeoffs.

\end{document}