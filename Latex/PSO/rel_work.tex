\section{Related work}
\label{sec:rel_work}

Due to the stochastic nature of the particle's path and the social interaction represented by the topology, the dynamics of the algorithm is hard to evaluate in general.
Plenty of works consider the ``stagnation'' that a particle is no longer able to find improvements in $ x^{G} $ and $ x^{P} $~\cite{Clerc06stagnationanalysis}.
Previous work that assumes stagnation can be categorized into two groups each based on how the analysis treats the stochastic factors.
The first approach is to ignore the stochastic factors.
Using this simplification, the convergence of a particle at stagnation can be analyzed~\cite{985692}. 
By building a linear system model~\cite{4424687}, the PSO algorithm can be viewed as a closed loop system and the convergence can be analyzed.
Based on such a convergence analysis, parameters can be set for best effect \cite{Trelea2003317}.

The second approach for handling the stochastic factors is based on the stochastic analysis.
By taking the mean of the stochastic variables, the stochastic terms can be converted into
constant terms.
A convergence analysis of the mean and variance of a particle at stagnation can also be obtained using the characteristic equation in a discrete-time model 
\cite{Jiang20078}.
In a similar way, 
other moments can be computed
\cite{5175367,Poli:2007:EAS:1276958.1276977,Poli:2008:DSS:1384929.1384944}.
Using the discrete-time system model of different moments, the equilibrium can be found.
The stability requirements can be obtained from the norm by setting the root values of the characteristic equation to all be less than 1.

In order to model the stochastic behavior of the particle swarm optimization, \cite{1637686} models the stochastic factor as a bounded nonlinear feedback.
The global best and the personal best are integrated into the nonlinear feedback, which depends an assumption of the constant personal best and constant global best.
The analysis tool of $ l-2 $ stability has also been imported to understand the impact of the stochastic factors on the convergence~\cite{5160341}.
There is also some work that addresses the dynamics when a particle is not in the stagnation phase.
The discrete-time dynamics of PSO, that is, the dynamics of particle trajectory, can be approximated
using a continuous-time model
\cite{5675669}.
Furthermore, the probability of convergence in time can be analyzed by viewing the update process as a random search process~\cite{vandenBergh:2010:CPP:2010420.2010421}.
The process of particles reaching a local optimum has also been analyzed~\cite{Schmitt:2013:PSO:2463372.2463563}.

Most of the efforts are still made on analyzing the particle level.
The question of the swarm behavior has not been clearly and completely answered yet.

In this paper, we 