\section{Related work}
\label{sec:rel_work}

Although the input-to-state stable analysis given in this paper can be applied to many versions of PSO,
for this work we use the formulas from Kennedy's most recent definition of PSO\cite{4223164}.
This version of PSO includes a ring topology, and a constricted position update rule. 
The constricted position update rule is

\begin{subequations}
\label{eq:pso_alg}
\begin{equation}
\label{eq:up_vel}
\begin{aligned}
v_{ij}(k+1) & = \chi [ v_{ij}(k) 
 + \phi^{P} u^{P}_{ij}(k) (x^{P}_{ij}(k) - x_{ij}(k)) + \phi^{G} u^{G}_{ij}(k) ( x^{G}_{ij}(k) - x_{ij}(k)) ],
\end{aligned}
\end{equation}
\begin{equation}
\label{eq:up_pos}
x_{ij}(k+1) = x_{ij}(k) + v_{ij}(k+1).
\end{equation}
\end{subequations}
$ x_{ij}(k) $ represents the position of particle $ i $ in dimension $ j $ at time $ k $.
$ v_{ij}(k) $ similarly represents the velocity of particle $ i $ in dimension $ j $ also at time $ k $.
$ x^{G}_{ij}(k) $ and $ x^{P}_{ij}(k) $ are global (actually topology) and personal best positions observed by the swarm and the particle respectively. 
$ u^{G}_{ij}(k) $ and $ u^{P}_{ij}(k) $ are independent random values drawn from $ [0,1] $.
$ \chi \in ( 0, 1 ) $, $ \phi^{P} $ and $ \phi^{G} $ are algorithm parameters.

Due to the stochastic nature of the particle's path and the social interaction represented by the topology, the dynamics of the algorithm is hard to evaluate in general.
However once a particle is no longer able to find improvements in $ x^{G} $ and $ x^{P} $, it exhibits the \emph{stagnation phenomenon} \cite{Clerc06stagnationanalysis}.
In this state the analysis is easier since there is no effect from the topology.

Previous work that assumes stagnation can be categorized into two groups each based on how the analysis treats the stochastic factors.
The first approach is to ignore the stochastic factors.
Using this simplification, the convergence of a particle at stagnation can be analyzed \cite{985692}. 
By building a linear system model \cite{4424687}, the PSO algorithm can be viewed as a closed loop system and the convergence can be analyzed.
Based on such a convergence analysis, parameters can be set for best effect \cite{Trelea2003317}.

The second approach for handling the stochastic factors is based on the stochastic analysis.
By taking the mean of the stochastic variables, the stochastic terms can be converted into
constant terms.
A convergence analysis of the mean and variance of a particle at stagnation can also be obtained using the characteristic equation in a discrete-time model 
\cite{Jiang20078}.
In a similar way, 
other moments can be computed
\cite{5175367,Poli:2007:EAS:1276958.1276977,Poli:2008:DSS:1384929.1384944}.
Using the discrete-time system model of different moments, the equilibrium can be found.
The stability requirements can be obtained from the norm by setting the root values of the characteristic equation to all be less than 1.

There is also some work that addresses the dynamics when a particle is not in the stagnation phase.
The discrete-time dynamics of PSO, that is, the dynamics of particle trajectory, can be approximated
using a continuous-time model
\cite{5675669}.
Furthermore, the probability of convergence in time can be analyzed
by viewing the update process as a random search process
\cite{vandenBergh:2010:CPP:2010420.2010421}.
The process of particles reaching a local optimum
has also been analyzed
\cite{Schmitt:2013:PSO:2463372.2463563}.