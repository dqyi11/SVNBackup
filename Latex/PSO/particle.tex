\section{Particle analysis}
\label{sec:particle}

In order to understand how the particle converges to the global best, we let $ x^{G} $ be the reference position.
We have interest with three types of movement range, which are defined by three types of reference positions.
\begin{itemize}
\item \textbf{global best} $ \lVert x(k) - x^{G} \rVert $ indicates the distance to the global best, which depends on  $ \lVert x^{P}(k) - x^{G} \rVert $.
It is given in Equation \eqref{eq:rel_gb}.
\item \textbf{initial position} $ \lVert x(k) - x(0) \rVert $ indicates the distance to the initial position, which depends on $ \lVert x^{G} - x(0) \rVert $ and $ \lVert x^{P}(k) - x(0) \rVert $.
It is given in Equation \eqref{eq:rel_init}.
\item \textbf{global optimal} $ \lVert x(k) - x^{*} \rVert $ indicates the distance to the global optimal, which depends on $ \lVert x^{G} -  x^{*} \rVert $ and $ \lVert x^{P}(k) -  x^{*} \rVert $.
It is given in Equation \eqref{eq:rel_opt}.
\end{itemize}

\begin{equation}
\label{eq:rel_gb}
\begin{aligned}
\begin{bmatrix}
v(k+1) \\
x(k+1) - x^{G}
\end{bmatrix}
 = A(k) 
\begin{bmatrix}
v(k) \\
x(k) - x^{G}
\end{bmatrix}
+ B(k) 
\begin{bmatrix}
0 \\
x^{P}(k) - x^{G}
\end{bmatrix}
\end{aligned}
\end{equation}

\begin{equation}
\label{eq:rel_init}
\begin{aligned}
\begin{bmatrix}
v(k+1) \\
x(k+1) - x(0)
\end{bmatrix}
 = A(k) 
\begin{bmatrix}
v(k) \\
 x(k) - x(0)
\end{bmatrix}
+ B(k) 
\begin{bmatrix}
x^{G} - x(0) \\
x^{P}(k) - x(0)
\end{bmatrix}
\end{aligned}
\end{equation}

\begin{equation}
\label{eq:rel_opt}
\begin{aligned}
\begin{bmatrix}
v(k+1) \\
x(k+1) - x^{*}
\end{bmatrix}
 = A(k) 
\begin{bmatrix}
v(k) \\
x(k) - x^{*}
\end{bmatrix}
+ B(k) 
\begin{bmatrix}
x^{G} - x^{*} \\
x^{P}(k) - x^{*}
\end{bmatrix}
\end{aligned}
\end{equation}



\subsection{What happens in a single hill case}

When the fitness distribution is a single hill, the analysis can be easier.
In most of the cases, the particles should all converge into a single hill where the global best locates in.



It means that there is no effect from a personal best.p
[TODO] random walk reaching a point


\begin{mylem}
In a single hill case, when $ x^{G} = x^{*} $, the particle will converge to $ x^{G} $ if the particle is ISS.
\end{mylem}

\begin{mylem}
In a single hill case, if $ x^{G} \not = x^{*} $, a particle will never get to 
\begin{proof}
Consider the velocity $ v(k+1) $ consists of two parts, inertia $ \chi v(k) $ and attractive force $ \chi \phi^{P} (x^{P}(k) - x(k) ) + \chi \phi^{G} ( x^{G} - x(k) ) $.
While the particle moves into $ x^{G} $, the attractive force becomes zero but the inertia still exists, due to the velocity moves the particle into this current position.
\end{proof} 
\end{mylem}

\begin{mythm}
In a single hill case, the particle will almost surly find a $ \hat{x^{*}} $ that $ f(\hat{x^{*}}) > f(x^{G}) $ if $ f( x^{G} ) < f( x^{*}) $.
\begin{proof}
Because the particle cannot stop at $ x(k) = x^{G} = x^{P} $.
It will finally arrives into a region that $ f(x) > f(x^{G}) $.

As in Figure \ref{fig:categorize_regions}, the solution space will be divided into three types of regions by the global best and the personal best.
\begin{itemize}
\item $ f(x) > f(x^G) $
Once a particle gets into this region, it updates both global best and personal best. 
It becomes a leader of the swarm.
\item $ f(x^{G}) > f(x) > f(x^{P}) $
Once a particle gets into this region, it updates only the personal best.
The solution space is then re-divided.
\item $ f(x) < f(x^{P}) $
When a particle is in this region, it only moves as a random walk.
\end{itemize}

\begin{figure}
\centering
\includegraphics[width=0.7\linewidth]{./fig/categorize_regions}
\caption{How global best and personal best divide the solution space.}
\label{fig:categorize_regions}
\end{figure}

The result of the movement of a particle is determined by which region of the solution space it moves in.

\end{proof}
\end{mythm}

\begin{figure}[ht]
\centering
\includegraphics[width=.7\linewidth]{./simfig/case1/distribution1}
\label{fig:case1-1:distribution} 
\end{figure}

\begin{figure}[ht]
\centering
\includegraphics[width=.7\linewidth]{./simfig/case1/position1-1} 
\label{fig:case1-1:position}
\caption{$ \chi = 0.72984 , \phi^{P} = 2.05 , \phi^{G} = 2.05 $ }
\end{figure}

\begin{figure}[ht]
\centering
\includegraphics[width=.7\linewidth]{./simfig/case1/fitness1-1} 
\label{fig:case1-1:fitness}
\caption{$ \chi = 0.72984 , \phi^{P} = 2.05 , \phi^{G} = 2.05 $ }
\end{figure}

\begin{figure}[ht]
\centering
\includegraphics[width=.7\linewidth]{./simfig/case1/position1-2} 
\label{fig:case1-2:position}
\caption{$ \chi = 0.72984 , \phi^{P} = 2.05 , \phi^{G} = 2.05 $ }
\end{figure}
  
\begin{figure}[ht]
\centering
\includegraphics[width=.7\linewidth]{./simfig/case1/fitness1-2} 
\label{fig:case1-2:fitness} 
\caption{$ \chi = 0.72984 , \phi^{P} = 2.05 , \phi^{G} = 2.05 $ }
\end{figure}

Show the exploitation


\subsection{More than single hill case}

Show the exploration

In order to measure how likely the particle can moves to a position that $ f(x) > f(x^{G}) $

\begin{mylem}

\begin{proof}
By Markov's inequality, we have
\begin{equation}
P( \lVert x(k) - x^{*} \rVert \geq \lVert x^{G} - x^{*} \rVert ) \leq \frac{ E( \lVert x(k) - x^{*} \rVert ) }{ \lVert x^{G} - x^{*} \rVert }.
\end{equation} 
By the boundary, we have
\begin{equation}
E( \lVert x(k) - x^{*} \rVert ) \leq \delta ( \max ( \lVert x^{G} - x^{*} \rVert , \lVert x^{P}(k) - x^{*}  \rVert ) ),
\end{equation}
in which $ \delta () $ is the boundary function.
\begin{equation}
\begin{aligned}
& P( \lVert x(k) - x^{*} \rVert < \lVert x^{G} - x^{*} \rVert ) \\
= & 1 - P( \lVert x(k) - x^{*} \rVert \geq \lVert x^{G} - x^{*} \rVert ) \\
> & 1 - \frac{ E( \lVert x(k) - x^{*} \rVert ) }{ \lVert x^{G} - x^{*} \rVert } \\
> & 1 - \frac{ \delta ( \max ( \lVert x^{G} - x^{*} \rVert , \lVert x^{P}(k) - x^{*}  \rVert ) ) }{ \lVert x^{G} - x^{*} \rVert }.
\end{aligned}
\end{equation}
\end{proof}
\end{mylem}

\begin{figure}[ht]
\centering
\includegraphics[width=.7\linewidth]{./simfig/case2/distribution2}
\label{fig:case2-1:distribution} 
\end{figure}

\begin{figure}[ht]
\centering
\includegraphics[width=.7\linewidth]{./simfig/case2/position2-1} 
\label{fig:case2-1:position}
\caption{$ \chi = 0.72984 , \phi^{P} = 2.05 , \phi^{G} = 2.05 $ }
\end{figure}

\begin{figure}[ht]
\centering
\includegraphics[width=.7\linewidth]{./simfig/case2/fitness2-1} 
\label{fig:case2-1:fitness}
\caption{$ \chi = 0.72984 , \phi^{P} = 2.05 , \phi^{G} = 2.05 $ }
\end{figure}

\begin{figure}[ht]
\centering
\includegraphics[width=.7\linewidth]{./simfig/case2/position2-2} 
\label{fig:case2-2:position}
\caption{$ \chi = 0.72984 , \phi^{P} = 2.05 , \phi^{G} = 2.05 $ }
\end{figure}
  
\begin{figure}[ht]
\centering
\includegraphics[width=.7\linewidth]{./simfig/case2/fitness2-2} 
\label{fig:case2-2:fitness} 
\caption{$ \chi = 0.72984 , \phi^{P} = 2.05 , \phi^{G} = 2.05 $ }
\end{figure}

