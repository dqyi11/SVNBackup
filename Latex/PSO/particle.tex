\section{Particle analysis}
\label{sec:particle}


As in Figure \ref{fig:categorize_regions}, the solution space will be divided into three types of regions by the global best and the personal best.
\begin{itemize}
\item $ f(x) > f(x^G) $
Once a particle gets into this region, it updates both global best and personal best. 
It becomes a leader of the swarm.
\item $ f(x^{G}) > f(x) > f(x^{P}) $
Once a particle gets into this region, it updates only the personal best.
The solution space is then re-divided.
\item $ f(x) < f(x^{P}) $
When a particle is in this region, it only moves as a random walk.
\end{itemize}

\begin{figure}
\centering
\includegraphics[width=0.7\linewidth]{./fig/categorize_regions}
\caption{How global best and personal best divide the solution space.}
\label{fig:categorize_regions}
\end{figure}

The result of the movement of a particle is determined by which region of the solution space it moves in.

\subsection{What happens in a single hill case}

When the fitness distribution is a single hill, the analysis can be easier.
In most of the cases, the particles should all converge into a single hill where the global best locates in.

\begin{myprop}
In a single hill case, $ x^{p}(k) = x(k) $. 
\end{myprop}

It means that there is no effect from a personal best.p
[TODO] random walk reaching a point


\begin{mylem}
In a single hill case, when $ x^{G} = x^{*} $, the particle will converge to $ x^{G} $ if the particle is ISS.
\end{mylem}

\begin{mylen}
In a single hill case, when $ f( x^{G} ) < f( x^{*}) $, the particle will converge to $ \hat{x^{*}} $ and $ f(\hat{x^{*}}) \geq f(x^{G}) $ if the particle is ISS.
The probability of $ \hat{x^{*}} = x^{G} $ is .
\end{mylem}

\subsection{More than single hill case}




