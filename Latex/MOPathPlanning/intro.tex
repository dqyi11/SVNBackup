\section{Introduction}

Path planning is a fundamental problem in the autonomy of the mobile robot.
Commonly it branches into low-level planning and high-level planning.
Low-level planning considers how an action should be completed.
The planned path will be applied to trajectory following in a relatively high accuracy.
The dynamics of the actuator should be included in the planning process.
For example, how a robotic arm moves in the assembling some parts. 
High-level planning is usually used for generating task-level behaviors.
The target is having a sequence of waypoints that defines a motion pattern that satisfies the task requirement.

As the most natural way of human communications, the semantic commands contain the objectives for task-level planning.
In order to understand the human's purpose exactly, the adverb in the human's semantic command cannot be ignored.
Involving adverb in defining the objectives of a task leads to the multi-objective optimization.

The problem becomes how to place a sequence of waypoints to optimize several objectives.


