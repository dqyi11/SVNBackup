\documentclass[letterpaper, 10 pt, conference]{ieeeconf}

\IEEEoverridecommandlockouts                              % This command is only needed if 
                                                          % you want to use the \thanks command

\overrideIEEEmargins                                      % Needed to meet printer requirements.

\title{\LARGE \bf
Homotopy-based RRT$^{*}$
}

\author{
Daqing Yi, Michael A. Goodrich and Kevin D. Seppi
\thanks{Daqing Yi, Michael A. Goodrich and Kevin D. Seppi are with Department of Computer Science, Brigham Young University, Provo, UT, 84604, USA.
{\tt\small daqing.yi@byu.edu, mike@cs.byu.edu, kseppi@cs.byu.edu} }
}

\begin{document}


\maketitle
\thispagestyle{empty}
\pagestyle{empty}


%%%%%%%%%%%%%%%%%%%%%%%%%%%%%%%%%%%%%%%%%%%%%%%%%%%%%%%%%%%%%%%%%%%%%%%%%%%%%%%%
\begin{abstract}
The homotopy 


\end{abstract}


%%%%%%%%%%%%%%%%%%%%%%%%%%%%%%%%%%%%%%%%%%%%%%%%%%%%%%%%%%%%%%%%%%%%%%%%%%%%%%%%
\section{INTRODUCTION}
\label{sec:intro}

The most popular focus on the robotic path planning is optimality.
In planning the robot's motion in a task, the intent of the task supervisor is usually modeled into single objective~\cite{6974170} or multiple objective optimization~\cite{yi2014supporting}.
The generated path is a solution that optimizes the given objectives, for example, minimizing the Euclidean distance.
There exist plenty of other human intents that are hard to be modeled using measurable metrics.
A class of them can be relevant with the ``shape'' of the planned path, especially in environments with complex obstacles.
For example, the human supervisor would like the robot reaching a target in the shortest time while passing through some locations.
The most common solution is defining via points as a constraint for the path planning problem.
However, sometimes this intent is only a preference instead of a hard constraint.

\emph{Homotopy} is imported to model the inherent topological similarity of the paths.
Given two paths, if one can be deformed into the other without encroaching any obstacle, they are said to be \emph{homotopic}~\cite{Hernandez201544}.
This leads to a \emph{homotopy class}, in which any two paths are homotopic.
The homotopy of the paths can help to model the human intent, which can be applied to several scenarios.
\begin{itemize}
\item Human initialized homotopic constraint \\
If the task supervisor has a strong opinion on the topology of the path, the supervisor can define the homotopic constraint by either initializing a reference path or defining via-regions.
The problem becomes finding an optimal path of a given homotopy class~\cite{Hershberger199463}.
The homotopic constraint reduce the exploring space, which brings the efficiency enhancement.
\item Optimality of different homotopy classes \\
If the task supervisor has a weak opinion on the topology of the path, the planning algorithm can provide the best solution of each homotopy class to the supervisor.
The supervisor can then select one that satisfy his/her intent.
\end{itemize}



\section{Related work}
\label{sec:related_work}

\subsection{Homotopy-based Path Planning}

Homotopy-based path planning depends on the determination of the homotopy equivalence of two paths, which is usually computationally expensive.


By algebraic topology, the path of same homotopy class can be determined by using orientable band~\cite{Hershberger199463}.
Semi-algebraic cuts are used to converting the path into ``word'' so that the homotopic equivalence can be compared~\cite{Grigoriev:1998:PAS:281508.281528}.
The Cauchy Integral theorem has been introduced to determine homotopy class by marking the positions in the obstacles as undefined~\cite{AAAI101920}.

\subsection{Homotopy RRT / PRM}

By homotopic redundancy, the paths from PRM could be compared by the homotopy classes~\cite{1041613}.
By dividing the space using the rays crossing each obstacle, reference frames could be used to represent the paths into canonical sequences~\cite{Hernandez201544}.

\subsection{Bidirectional RRT}

Bidirectional RRT has been introduced to accelerate the exploration process while the optimality is still preserved~
\cite{Jordan.Perez.ea:CSAIL13}~\cite{starek2014bidirectional}.

\section{Algorithm}
\label{sec:algorithm}

\subsection{Homotopy-base space decomposition}

\subsection{Bidirection RRT$^{*}$}

\section{Application}
\label{sec:application}

\section{Analysis}
\label{sec:analysis}

\section{Conclusion}
\label{sec:conclusion}

\bibliographystyle{IEEEtran}
\bibliography{reference}

\end{document}