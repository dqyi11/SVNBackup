\documentclass[letterpaper, 10 pt, conference]{ieeeconf}

\IEEEoverridecommandlockouts                              % This command is only needed if 
                                                          % you want to use the \thanks command

\overrideIEEEmargins                                      % Needed to meet printer requirements.


\title{\LARGE \bf
Homotopy RRT
}

\author{
Daqing Yi
\thanks{Daqing Yi is with Department of Computer Science, Brigham Young University, Provo, UT, 84604, USA.
{\tt\small daqing.yi@byu.edu} }
}


\begin{document}


\maketitle
\thispagestyle{empty}
\pagestyle{empty}


%%%%%%%%%%%%%%%%%%%%%%%%%%%%%%%%%%%%%%%%%%%%%%%%%%%%%%%%%%%%%%%%%%%%%%%%%%%%%%%%
\begin{abstract}



\end{abstract}


%%%%%%%%%%%%%%%%%%%%%%%%%%%%%%%%%%%%%%%%%%%%%%%%%%%%%%%%%%%%%%%%%%%%%%%%%%%%%%%%
\section{INTRODUCTION}
\label{sec:intro}

The most popular focus on the robotic path planning is optimality.
Given a measurable metric, the planned path would be a optimized solution to the given objective.
This has been widely applied to model the intent of the task supervisor.
For example, if the task is reaching a target in shortest time, the cost is defined as Euclidean distance and the planned path should be the shortest.
However, there also exist plenty of intents that are hard to be measured by any metric, which are usually relevant with the topology.
For example, though the execution time is expected to be minimized, it would be better if the robot could pass some specific location.
It indicates that sometimes the ``shape'' of the paths could be an important factor to consider in planning, especially in environments with complex obstacles.




\bibliographystyle{IEEEtran}
\bibliography{reference}

\end{document}