\section{Stepwise uncertainty reduction}

\begin{itemize}
\item [\textbf{2013}] Multiobjective optimization using Gaussian process emulators via
stepwise uncertainty reduction~\cite{picheny2013multiobjective}
\item [\textbf{2014}] A stepwise uncertainty reduction approach to constrained global optimization~\cite{picheny2014stepwise}
\end{itemize}

\emph{Stepwise uncertainty reduction} (SUR strategy) is used to reduce an uncertainty measure by sequential sampling about a quantity of interest.
The objective space is decomposed into cells, which are hyper-rectangles $ \Omega_{i} $.
A set of non-dominated cells $ I^{*} $ can be found by comparing with all other cells.

\emph{Improvement} is defined as the difference between the current observed minimum and the new function value if it is positive, or zero otherwise, which is written as $ \max(0, u^{min}_{n}-Y(x)) , y^{min}_{n} = \min(y_{1}, \cdots , y_{n}) $.
\emph{Expected improvement} is conditional expectation under the GP model
$ EI(x) = E[ \max(0, u^{min}_{n}-Y(x)) \mid A_{n} ] $.
The new measurement is selected as $ x_{n+1} = \arg \max_{x \in X} EI(x) $.