\documentclass[10pt,a4paper]{article}
\usepackage[latin1]{inputenc}
\usepackage{amsmath}
\usepackage{amsfonts}
\usepackage{amssymb}
\usepackage{graphicx}
\begin{document}

\title{Dynamics of Particle Swarm}

\date{}

\maketitle

In order to understand what happens in a PSO(Particle swarm optimization), people are usually interested with the ``exploration and exploitation'' of this algorithm.
\begin{itemize}
\item ``Exploration'' means \textbf{search capability}, which determines how likely the particles could find local best and global best;
\item ``Exploitation'' leads to \textbf{convergence}, which shows how the particles utilize the found current best.
It also tells what happens on the swarm behavior when no new global best can be found.
\end{itemize}

There is no guarantee that PSO could always find a global optimal.
It is not hard to give an example that the global optimal could not be found. 
The factors impact the likelihood that the global optimal is found include:
\begin{itemize}
\item number of particles,
\item $ \chi $, $ \Phi^{P} $, $ \Phi^{G} $,
\item and the distribution of the fitness space.
\end{itemize}

Our analysis will focus on a standard PSO.
It includes the canonical update rule and a ring topology (single and consistent global best).




\section{How a particle is led to a local best}

This depends on the shape of fitness space

\section{What happens when the global best is not changed}

Make the global best as the leader

\begin{itemize}
\item 
\item 
\end{itemize}


\end{document}