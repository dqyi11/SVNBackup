% LLNCStmpl.tex
% Template file to use for LLNCS papers prepared in LaTeX
%websites for more information: http://www.springer.com
%http://www.springer.com/lncs

\documentclass{llncs}
%Use this line instead if you want to use running heads (i.e. headers on each page):
%\documentclass[runningheads]{llncs}


\begin{document}
\title{Distributed Cooperative Attitude Synchronization and Tracking for Multiple Rigid Bodies}

%If you're using runningheads you can add an abreviated title for the running head on odd pages using the following
%\titlerunning{abreviated title goes here}
%and an alternative title for the table of contents:
%\toctitle{table of contents title}

\subtitle{EC EN 774 Project Proposal}

%For a single author
%\author{Author Name}

%For multiple authors:
\author{Daqing Yi}


%If using runnningheads you can abbreviate the author name on even pages:
%\authorrunning{abbreviated author name}
%and you can change the author name in the table of contents
%\tocauthor{enhanced author name}

%For a single institute
\institute{\email{dqyi11@gmail.com}}

% If authors are from different institutes 
%\institute{First Institute Name \email{email address} \and Second Institute Name\thanks{Thank you to...} \email{email address}}

%to remove your email just remove '\email{email address}'
% you can also remove the thanks footnote by removing '\thanks{Thank you to...}'


\maketitle

%\begin{abstract}
%abstract text goes here - Lorem ipsum dolor sit amet, consectetur adipiscing elit, sed do eiusmod tempor incididunt ut labore et dolore magna aliqua.
%\end{abstract}

This project refers to Dr. Ren's paper of ``Distributed Cooperative Attitude Synchronization and Tracking for Multiple Rigid Bodies'' \cite{5229134}.

In the network synchronization problem, though the interaction topology of multiple agents indicates a linear system model, the network model becomes nonlinear when the dynamics of each agent is nonlinear.
This project focuses on how the theoretical methods on nonlinear system can be applied to network synchronization problems on nonlinear models.

In \cite{5229134}, distributed cooperative attitude synchronization problem is discussed.
The attitude of a rigid body follows a nonlinear dynamics.
Thus, Dr. Ren models the dynamics of the network into a nonlinear system by representing each node with MRP.

Dr. Ren proposed a control law in a passivity approach.
By proposing a Lyapunov function and using \emph{LaSalle's invariance principle}, Dr. Wen proves that all the agents can converge to the reference attitude without angular velocity measurement through network synchronization.

The problem is extended to time-varying reference tracking.
A virtual leader is imported as a node in the topology, which connects with only a portion of the agents.
Because the virtual leader only broadcasts the reference in the interaction, the agents shows reference tracking due to the convergence property in network synchronization.
Dr. Ren uses a Lyapunov function and \emph{Barbalat's lemma} to prove that the dynamics of the agents converge to the time-varying reference when there is a directed spanning tree in the topology. 

The project plan consists of 
\begin{itemize}
\item recognizing the nonlinear model of rigid body attitude dynamics;
\item understanding how the cooperative attitude synchronization problem is formed;
\item listing the properties and assumptions in the problem definitions;
\item understanding the proof strategies;
\item verifying the performance by simulation;
\item analyzing and discussing how the work can be enhanced.
\end{itemize}

%The bibliography, done here without a bib file
%This is the old BibTeX style for use with llncs.cls
\bibliographystyle{splncs}

%Alternative bibliography styles:
%the following does the same as above except with alphabetic sorting
%\bibliographystyle{splncs_srt}
%the following is the current LNCS BibTex with alphabetic sorting
%\bibliographystyle{splncs03}
%If you want to use a different BibTex style include [oribibl] in the document class line

\bibliography{reference}

\end{document}

