\documentclass[12pt]{article}

% packages

%\usepackage{times} % alt: cmbright
\usepackage[top=1in, bottom=1in, left=1in, right=1in]{geometry}
\usepackage{natbib}
\usepackage{amsmath}
\usepackage{amssymb}
\usepackage{latexsym}
\usepackage{sectsty}
\usepackage{amsfonts}
\usepackage{epsfig}
\usepackage{url}
\usepackage{microtype}
\usepackage{fixmath}
\usepackage{hyperref}
\usepackage{amsthm}
\usepackage{subfigure}
\usepackage{float}
\usepackage{hyperref}

\newtheorem{lem}{Lemma}
\newtheorem{defn}{Assumption}
\newtheorem{propty}{Property}
\newtheorem{thm}{Theorem}

% references

\newcommand{\mysec}[1]{Section~\ref{sec:#1}}
\newcommand{\myapp}[1]{Appendix~\ref{app:#1}}
\newcommand{\myeq}[1]{Equation~\ref{eq:#1}}
\newcommand{\myeqp}[1]{Eq.~\ref{eq:#1}}
\newcommand{\mychap}[1]{Chapter~\ref{chap:#1}}
\newcommand{\myfig}[1]{Figure~\ref{fig:#1}}

% math conveniences

\newcommand{\g}{\,\vert\,}
\newcommand{\E}{\textrm{E}}
\newcommand{\vct}[1]{\textbf{#1}}
\newcommand{\realline}{\mathbb{R}}
\newcommand{\indpt}{\protect\mathpalette{\protect\independenT}{\perp}}
\def\independenT#1#2{\mathrel{\rlap{$#1#2$}\mkern2mu{#1#2}}}
\newcommand{\h}[1]{\textrm{H}\left( #1 \right)}
\newcommand{\half}{\frac{1}{2}}
\newcommand{\new}{\textrm{new}}

\newcommand{\mult}{\textrm{Mult}}
\newcommand{\dir}{\textrm{Dir}}
\newcommand{\discrete}{\textrm{Discrete}}
\newcommand{\Bern}{\textrm{Bern}}
\newcommand{\DP}{\textrm{DP}}
\newcommand{\GP}{\textrm{GP}}
\newcommand{\Bet}{\textrm{Beta}}

% paragraph spacing

\setlength{\parindent}{0pt}
\setlength{\parskip}{2ex plus 0.5ex minus 0.2ex}

\allsectionsfont{\sffamily\mdseries}
\paragraphfont{\sffamily\bfseries}

\usepackage{algorithm}
\usepackage{algorithmic}
\renewcommand{\algorithmicrequire}{\textbf{Input:}}
\renewcommand{\algorithmicensure}{\textbf{Output:}}


\begin{document}

\title{\textsf{Survey on Multi-Objective Path Planning}}
\author{\textsf{Daqing Yi}}
\date{\textsf{}}

\maketitle

\paragraph{\textbf{Tarbiat Modares University, Tehran, Iran}}
\cite{masehian2010multi2} \cite{masehian2010multi} use PSO to solve a path planning problem, in which they model the problem with two objectives ``shortest path'' and ``smoothest path''. A fitness function is defined as scalarization of two objective functions, which is $ \mbox{fitness} = \lambda_{1} F_{1} + \lambda_{2} F_{2} $. This forms a global planning process, which gives next target position. If there is a directly connecting line intersects with any obstacle, a local planning process using probabilistic roadmap method(PRM) is imported to search shortest path between two positions. 

\paragraph{\textbf{Tijuana Institute of Technology Tijuana, Mexico}}
\cite{castillo2007multiple} uses Pareto optimality to rank the populations generated in a Genetic Algorithm. The world has been modelled into a grid system, so that each path can be encoded into a chromosome expression.

\paragraph{\textbf{Tel-Aviv University, Israel}}
\cite{avigad2009interactive} 

\cite{moshaiov2004concept}

\cite{moshaiov2007extended} 

\cite{avigad2007sequential}

\paragraph{\textbf{University of Salford, UK}}
\cite{soltani2004fuzzy} imports a path planning problem on site construction, which searches for a balance between path length, safety and visibility. Fuzzy membership funtions are imported to define objectives on safety and visibility. Including the objective of minimum transportation cost, a composite single objective function is obtained by linear combination of three objectives. The problem solver used is a Dijkstra search.

\paragraph{\textbf{IIT Kanpur, Kanpur, India}}
\cite{ahmed2011multi} form a path planning problem which expects to reaching a goal point in shortest time while generate safe paths. They define three forms of safety measure:
\begin{itemize}
\item difficulty level by artificial potential field;
\item minimizing total obstacles in neighborhood to avoid obstacle cluttered areas;
\item maximizing visibility (mean sensor field of view) using isovists lines.
\end{itemize}
A map has been discretized for GA encoding purpose. Using NSGA-II in \cite{deb2002fast}, a sequence of waypoints are generated in considering those metrics above. Importing either B-spline or PP-spline, a complete path in continuous space can be generated with the consideration on smoothness.

\paragraph{\textbf{Queensland University of Technology, Australia}}
\cite{wu2007fuzzy} \cite{wu2011multi}

\paragraph{\textbf{Air Force Institute of Technology, USA}}
\cite{Pohl:2008:MUM:1516744.1516965}

\paragraph{\textbf{École Centrale de Nantes, France}}
\cite{ur2010multi}

\paragraph{\textbf{McGill University, Canada}}
\cite{higuera2012socially}

\bibliographystyle{apalike}
\bibliography{reference}

\end{document}
