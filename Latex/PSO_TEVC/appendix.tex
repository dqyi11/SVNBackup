\section*{Appendix}

\subsection{ISS-Lyapunov function}
\label{sec:iss_lyapunov:func}

Using the definitions of a $ K $-function and a $ KL $-function in Section \ref{sec:system}
we can define a ISS-Lyapunov function as follows,
%\begin{itemize}
%\item 
an ISS-Lyapunov function $ V : \mathbb{R}^{n} \rightarrow \mathbb{R}_{\geq 0} $ satisfies:
\begin{enumerate}
\item $ \exists \alpha_{1}, \alpha_{2} \in \mathbb{K} $ such that 
$ \forall \xi \in \mathbb{R}^{n} $ , $ \alpha_{1} ( | \xi | ) \leq V( \xi ) \leq \alpha_{2}  ( | \xi | ) $.
\item $ \exists \alpha_{3} \in \mathbb{K}_{\infty} , \sigma \in \mathbb{K} $ such that $ \forall \xi \in \mathbb{R}^{n}, \forall \mu \in \mathbb{R}^{m} $,$  V( f( \xi, \mu ) ) - V( \xi ) \leq - \alpha_{3} ( | \xi | ) + \sigma ( | \mu | ) $. 
\end{enumerate}

\subsection{Proof of Theorem \ref{thm:iss}}
\label{sec:thm:iss:proof}
		
\begin{proof} 
Let $ P $ be an identity matrix.
As $ | \lambda_{\max} ( A(k) ) | < 1 $, we have
%\begin{equation}
%\nonumber
$
\lVert A^{T}(k) P A(k) \rVert \leq \lVert P \rVert \lVert A(k) \rVert^{2} \leq \lVert P \rVert | \lambda_{\max} ( A(k) ) |^{2} $ $  <  \lVert P \rVert .
$
%\end{equation}
Because $ P $ is an identity matrix it is positive definite, and thus $ A^{T}(k) P A(k) $ is positive definite or positive semi-definite by definition.
So by positive definite ordering we have $ A^{T}(k) P A(k) < P $.
		
Let $ -Q(k) = A^{T}(k) P A(k) - P $. Since $ A^{T}(k) P A(k) < P $ then $ - Q(k) < 0 $ furthermore $ \exists Q' \forall k, Q(k) > Q' > 0 $. 
		
By the Lemma 3.5 in \cite{Jiang2001857}, if we can show that a proposed positive definite Lyapunov function is an ISS-Lyapunov function, then the system is ISS.
		
Define a Lyapunov function
\begin{equation}
\label{eq:lyapunov_v}
V( X(k) ) = X^{T} (k) P X(k).
\end{equation}
We can have
$
\lambda_{min}(P) | X(k) |^{2} \leq V( X(k) )\leq \lambda_{max}(P) | X(k) |^{2}
$ and $ \lambda_{min}(P) = \lambda_{max}(P) $.
		
Let $ \alpha_{1} ( \xi )= \lambda_{min} \xi^{2} $
and 
$ \alpha_{2} ( \xi )= \lambda_{max} \xi^{2} $,
we have $ V(x) $ satisfying condition 1 of the ISS-Lyapunov function definition.
		
By applying equation \eqref{eq:pso_up_linalg_simp} to $ V( X(k+1) ) - V( X(k) ) $, we have
\begin{equation}
\label{eq:lyapunov_delta2}
\begin{aligned}
& V( X(k+1) ) - V( X(k) ) \\
%	= & - X^{T}(k) [ A^{T}(k) P A(k) - P ] X(k) + 2 X^{T}(k)  A^{T}(k) P B(k) U(k) \\
%	& + U^{T}(k) B^{T}(k) P B(k) U(k) \\
%	\leq & - X^{T}(k) Q' X(k) + 2 X^{T}(k)  A^{T}(k) P B(k) U(k) \\
%	& + U^{T}(k) B^{T}(k) P B(k) U(k) \\
%	\leq & - \lambda_{min}(Q') | X(k) |^{2} + 2  \lVert A^{T}(k) P B(k) \rVert  U(k) | | X(k) | \\
%	& + \lVert B^{T}(k) P B(k) \rVert | U(k) |^{2}.
= & [ X^{T}(k)  A^{T}(k) + U^{T}(k) B^{T}(k) ] P [ A(k) X(k) + B(k) U(k) ] \\ & - X^{T}(k) P X(k) \\
= & X^{T}(k)  A^{T}(k) P A(k) X(k) +  X^{T}(k)  A^{T}(k) P B(k) U(k) \\
& + U^{T}(k) B^{T}(k) P A(k) X(k) + U^{T}(k) B^{T}(k) P B(k) U(k) \\ & - X^{T}(k) P X(k) \\
\end{aligned}
\end{equation}
As $ P $ is identity matrix, it is symmetric, thus
\begin{equation}
[ X^{T}(k)  A^{T}(k) P B(k) U(k) ]^{T} =  U^{T}(k) B^{T}(k) P A(k) X(k).
\end{equation}
$ V( X(k+1) ) , V( X(k) ) \in \mathbb{R} $, 
we have $ X^{T}(k)  A^{T}(k) P B(k) U(k) $ and $  U^{T}(k) B^{T}(k) P A(k) X(k) $ are both real value (like $ 1 \times 1 $ matrix).
Thus, 
\begin{equation}
 X^{T}(k)  A^{T}(k) P B(k) U(k) =   U^{T}(k) B^{T}(k) P A(k) X(k) .
\end{equation}
We then have
\begin{equation}
\label{eq:lyapunov_delta3}
\begin{aligned}
& V( X(k+1) ) - V( X(k) ) \\
= & - X^{T}(k) [ A^{T}(k) P A(k) - P ] X(k) \\
& + U^{T}(k) B^{T}(k) P B(k) U(k)  \\
& + 2 X^{T}(k)  A^{T}(k) P B(k) U(k) \\
\leq & - X^{T}(k) Q' X(k)  + U^{T}(k) B^{T}(k) P B(k) U(k) \\
& + 2 X^{T}(k)  A^{T}(k) P B(k) U(k) \\
\end{aligned}
\end{equation}

By applying matrix norm, we have
\begin{equation}
\begin{aligned}
& V( X(k+1) ) - V( X(k) ) \\
\leq & - \lambda_{min}(Q') | X(k) |^{2}  + | B^{T}(k) P B(k) | | U(k) |^{2} \\
& + 2  | A^{T}(k) P B(k) | | U(k) | | X(k) | \\
= & - \frac{1}{2} \lambda_{min}(Q') | X(k) |^{2} + | B^{T}(k) P B(k) | | U(k) |^{2} \\
& - \frac{1}{2} \lambda_{min}(Q') | X(k) |^{2} + 2  | A^{T}(k) P B(k) | | U(k) | | X(k) |  \\
= & - \frac{1}{2} \lambda_{min}(Q') | X(k) |^{2} \\
& + \left( \frac{2 | A^{T}(k) P B(k) |^{2}}{ ( \lambda_{min}(Q') )^{2} } + | B^{T}(k) P B(k) |  \right) | U(k) |^{2} \\
& - \frac{1}{2} \lambda_{min}(Q') [ | X(k) |^{2} - \frac{4 | A^{T}(k) P B(k) | }{ \lambda_{min}(Q') }  | X(k) | | U(k) | \\
& + \frac{4 | A^{T}(k) P B(k) |^{2}}{ ( \lambda_{min}(Q') )^{2} } | U(k) |^{2} ] \\
\end{aligned}
\end{equation}
		
By completing the square, we have
\begin{equation}
\label{eq:lyapunov_delta4}
\begin{aligned}
& V( X(k+1) ) - V( X(k) ) \\
= & - \frac{1}{2} \lambda_{min}(Q') | X(k) |^{2} \\
& + \left( \frac{2 | A^{T}(k) P B(k) |^{2}}{ ( \lambda_{min}(Q') )^{2} } + | B^{T}(k) P B(k) | \right) | U(k) |^{2} \\
& - \frac{1}{2} \lambda_{min}(Q') \left( | X(k) | - \frac{2 | A^{T}(k) P B(k) | }{ \lambda_{min}(Q') } | U(k) | \right)^{2} \\
	\leq & - \frac{1}{2} \lambda_{min}(Q') | X(k) |^{2} \\
 &	+ \left( \frac{2 \lVert A^{T}(k) P B(k) \rVert^{2}}{ ( \lambda_{min}(Q') )^{2} } 
	 + \lVert B^{T}(k) P B(k) \rVert \right) | U(k) |^{2}. 
\end{aligned}
\end{equation}
		
Because $ u^{P}(k) \in [0, 1] $, there exist an $ A' $ and $ B' $ such that $ \lVert A(k) \rVert \leq \lVert A' \rVert $ and $ \lVert B(k) \rVert \leq \lVert B' \rVert $.
We have $ \lVert A^{T}(k) P B(k) \rVert $ $  \leq \lVert A' \rVert \lVert P \rVert \lVert B' \rVert $ and $ \lVert B^{T}(k) P B(k) \rVert \leq \lVert P \rVert \lVert B' \rVert^{2} $.
		
Since the identity matrix $ P $ has $ || P || = 1 $:
\begin{equation}
\label{eq:lyapunov_delta5}
\begin{aligned}
& V( X(k+1) ) - V( X(k) ) \\
	\leq & - \frac{1}{2} \lambda_{min}(Q') | X(k) |^{2} + \left( \frac{2 \lVert A' \rVert^{2} \lVert B' \rVert^{2}}{ ( \lambda_{min}(Q') )^{2} } + \lVert B' \rVert^{2} \right) | U(k) |^{2}.
\end{aligned}
\end{equation}
		
Let
\begin{equation}
\nonumber
\alpha_{3} ( \xi )= \frac{1}{2} \lambda_{min}(Q') \xi^{2} ,
\end{equation}
and
\begin{equation}
\nonumber
\sigma ( \xi ) = \left( \frac{2 \lVert A' \rVert^{2} \lVert B' \rVert^{2}}{ ( \lambda_{min}(Q') )^{2} } +  \lVert B' \rVert^{2} \right) \xi^{2} .
\end{equation} 
Thus we have $  V( X(k+1) ) - V( X(k) ) $ satisfying condition 2 of the ISS-Lyapunov function definition and
so \eqref{eq:lyapunov_v} is an ISS-Lyapunov function.
Using Jiang's Lemma 3.5\cite{Jiang2001857}, the position-update component of PSO (equation \eqref{eq:pso_up_linalg_simp}) is input-to-state stable.
\end{proof}

\subsection{Proof of Corollary \ref{coro:param_unit_disc}}
\label{sec:coro:param_unit_disc:proof}

\begin{proof}
Let $ a = (1 + \chi) - \chi \phi $. 
The eigenvalues of $ A(k) $ are
\begin{equation}
\nonumber
 \lambda = \frac{ a \pm \sqrt{ a^{2} - 4 \chi } }{2} .
\end{equation}
There can be two cases.		
\begin{enumerate}
\item If $ a^{2} \geq 4 \chi $, the eigenvalues are complex number.
We have $ a \geq 2 \sqrt{\chi} $ or $ a \leq - 2 \sqrt{\chi} $.
			
If $ a \geq 2 \sqrt{\chi} $, then $ | \lambda_{\max} | < 1 $ derives 
\begin{equation}
\nonumber
0 < \frac{a-\sqrt{a^{2}-4\chi}}{2} \leq \frac{a+\sqrt{a^{2}-4\chi}}{2} < 1 .
\end{equation}
It means that $ 2 \sqrt{ \chi } \leq a < 1 + \chi $.
			
If $ a \leq 2 \sqrt{\chi} $, then $ | \lambda_{\max} | < 1 $ derives
\begin{equation}
\nonumber
-1 < \frac{a-\sqrt{a^{2}-4\chi}}{2} \leq \frac{a+\sqrt{a^{2}-4\chi}}{2} < 0 .
\end{equation}
It means that $ - (\chi+1) < a \leq - 2 \sqrt{\chi} $.
			
\item If $ a^{2} < 4 \chi $, the eigenvalues are real number.
We have $ - 2 \sqrt{\chi} < a < 2 \sqrt{\chi} $.
			
$ | \lambda_{\max} | < 1 $ derives
\begin{equation}
\nonumber
\frac{ a^{2} }{4} + \frac{ a^{2} - 4\chi }{4} < 1 .
\end{equation}
It means that $ - 2 \sqrt{ 2(1+\chi) } < a < 2 \sqrt{ 2(1+\chi) } $.
Because $ \sqrt{ 2(1+\chi) } > 2 \sqrt{ \chi } $, we have $ - 2 \sqrt{\chi} < a < 2 \sqrt{\chi} $.
\end{enumerate}
Combining these two cases, we have  $ - (1 + \chi) < a < 1 + \chi $.
It equals to $ \phi \in \left( 0 , \frac{2(1+\chi)}{\chi} \right) $.
\end{proof}	

\subsection{Proof of the Theorem \ref{thm:state_bound}}	
\label{sec:thm:state_bound:proof}

\begin{proof}
As we have the update equation as
$ X(k+1) = A(k) X(k) + B(k) U(k) $, we can derive 
\begin{equation}
X(k+1) = ( \prod_{k}^{i=0} A(i) ) X(0) + \sum_{i=0}^{k} [ ( \prod_{j=0}^{i-1} A(j) ) B(i) U(i)  ] 
\end{equation}
by recursively applying it.

By the property of matrix norm, we have
\begin{equation}
| X(k+1) | \leq ( \prod_{i=0}^{k} \lVert A(i) \rVert ) | X(0) | + \sum_{i=0}^{k} [ ( \prod_{j=0}^{i-1} \lVert A(j) \rVert ) \lVert B(i) \rVert | U(i) |  ].
\end{equation}

$ \forall i \in [0, k] $, let $ \lVert A(i) \rVert \leq \lVert A \rVert $, $  \lVert B(i) \rVert \leq \lVert B \rVert $ and $ | U(k) | = [ x^{G}(k) - x^{R}, x^{P}(k) - x^{R} ]^{T} $, we have
\begin{equation}
\label{eq:bound:final}
\begin{aligned}
& |  x(k+1) - x^{R} | \leq | X(k+1) | \\
& \leq ( \lVert A \rVert )^{k+1} | X(0) | + \sum_{i=0}^{k} [ ( \lVert A \rVert )^{i} \lVert B \rVert | U(i) |  ] \\
& = ( \lVert A \rVert )^{k+1} | X(0) | + \frac{1 - ( \lVert A \rVert )^{k+1} }{1 - \lVert A \rVert }  \lVert B \rVert | U(i) |
\end{aligned}
\end{equation}

%The boundary will be a function of $ bound ( \lVert A \rVert, \lVert B \rVert, | X(0) |, | U |, k ) $.
%Thus the minimum boundary is $ \min_{k} bound ( \lVert A \rVert, \lVert B \rVert, | X(0) |, | U |, k ) $.
%When we have $ \lVert A \rVert < 1 $, 
%$ ( \lVert A \rVert )^{k+1} \rightarrow 0 $ and
%$ \frac{1 - (\lVert A \rVert )^{k+1} }{1 - \lVert A \rVert} \rightarrow \frac{1}{1 - \lVert A \rVert } $
%as $ k \rightarrow \infty $.

$ ( \lVert A \rVert )^{k+1} $ shows the decay term and $ \frac{1 - ( \lVert A \rVert )^{k+1} }{1 - \lVert A \rVert }  \lVert B \rVert $ makes the boundary function $ \gamma () $.
\end{proof}


\subsection{Mean model of the position update component}
\label{app:mean_pso}

By applying mean to \eqref{eq:pso_up_linalg_simp}, we can have the mean of the position update component as
\begin{equation}
\label{eq:pso_up_linalg_simp:mean}
E( X(k+1) ) = A(k) E( X(k) ) + B(k) E( U(k) )
\end{equation}
with
$ A(k) = \begin{bmatrix}
\chi & - \chi \phi^{G}/2 - \chi \phi^{P}/2
\\ 
\chi & 1 - \chi \phi^{G}/2 - \chi \phi^{P}/2
\end{bmatrix} $
and
$ B(k) = \begin{bmatrix}
\chi \phi^{G}/2 & \chi \phi^{P}/2
\\ 
\chi \phi^{G}/2 & \chi \phi^{P}/2
\end{bmatrix} $.

$ E( X(k) ) = [ E( v(k) ), E( x(k) - x^{R} ) ]^{T} $ and $ E( U(k) ) = [ E( x^{G}(k) - x^{R} ) , E( x^{P}(k) - x^{R} ) ]^{T} $.

By taking $ x^{R} = x^{*} $, we can have \eqref{eq:mean:opt_bound}.

\subsection{Proof of Corollary \ref{coro:param_unit_disc:mean}}
\label{sec:coro:param_unit_disc:proof:mean}

\begin{proof}
The proof is similar with that in Subsection \ref{sec:coro:param_unit_disc:proof}.
In this case, $ a = (1 + \chi) - \frac{ \phi }{2} \chi $.
Similarly, we can have two cases and derive
$ - (1 + \chi) < a < 1 + \chi $.
It equals to 
$
\phi \in \left( 0 , \frac{4(1+\chi)}{\chi} \right) .
$
\end{proof}