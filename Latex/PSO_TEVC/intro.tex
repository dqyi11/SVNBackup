\section{Introduction}
\label{sec:introduction}

Particle swarm optimization has been a popular computational method that could efficiently find the solutions to given optimization problems, even when the gradient of the objective function is not available.
It is inspired from the social behavior of the flock of birds and the school of fish.
Therefore, algorithmic interpretations are needed to evaluate and support the PSO algorithm. 
There are a few research work that try to interpret why the PSO could effectively find the ``optimal'' solutions, when the PSO fails to find the optimal solutions and how the convergence process is like.

\emph{Stagnation} is one important phenomenon in understanding the search process of the evolutionary optimization algorithms.
It is commonly defined that the algorithm stops finding a better solution.
In the PSO, it means that all the particles in the swarm cannot find a better solution than the current global best.
Plenty of works consider the ``stagnation'' that a particle is no longer able to find improvements in $ x^{G} $ and $ x^{P} $~\cite{Clerc06stagnationanalysis}.
It means that the personal best and global best keep constant over some time steps \cite{4223160}.
Understanding the dynamics of the particles in stagnation is very helpful in analyzing the behavior and the optimization search capability of the particle swarm optimization.
Plenty of work have focused on the behavior of a particle when this happens, particularly where it converges to~\cite{Schmitt:2013:PSO:2463372.2463563,Poli:2008:DSS:1384929.1384944}.

If we decompose the search process of the PSO by the times of the global-best update, we can see that in each span the global best is a constant.
It is known that the PSO can usually converge very fast.
There could be a long time that the global best is constant, although the personal best of each particle is being frequently updated in each iteration.
Define the time span that the global best is unchanged as the \emph{global-best stagnation}.
The process of the PSO consists of a sequence of the global-best stagnations, in which the global best is refined in a newer global-best stagnation.

The information exchange among the particles depends on the update of the global best.
Thus in the global-best stagnation, the particles have no interaction between each other and they become a collection of independent agents.
Then the particles are in a competition relationship.
A particle that finds a better solution than the current global best updates the global best and transits the swarm into a new global-best stagnation.

If we view the movement of a particle is driven by a force from the personal best and a force from the global best, 
The stochastic terms in how a particle updates its position determines that a particle won't stop moving unless the global best and the personal best are the same.
Thus, we can have Property \ref{prop:unconverge_neq_gb}.
\begin{myprop}
\label{prop:unconverge_neq_gb}
A particle won't stop moving when its personal best and global best are not the same, 
which means that 
$ \lim_{k \rightarrow \infty} v(k) \neq 0 $, if $ x^{P} \neq x^{G} $.
\end{myprop}
It indicates that the import of the stochastic factors prevents the existence of an equilibrium position for a particle before its global best and personal best reach a consensus.
It means that the conventional analyses of point convergence methods are not enough to understand the dynamics of the optimization process in PSO.
In this paper, we will analyze the region convergence in the particle swarm optimization.