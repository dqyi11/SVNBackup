%%% Template originaly created by Karol Kozioł (mail@karol-koziol.net) and modified for ShareLaTeX use

\documentclass[a4paper,11pt]{article}

\usepackage[T1]{fontenc}
\usepackage[utf8]{inputenc}
\usepackage{graphicx}
\usepackage{xcolor}

\usepackage{tgtermes}

\usepackage[
pdftitle={CS660 Final Answer sheet}, 
pdfauthor={Daqing Yi, Brigham Young University},
colorlinks=true,linkcolor=blue,urlcolor=blue,citecolor=blue,bookmarks=true,
bookmarksopenlevel=2]{hyperref}
\usepackage{amsmath,amssymb,amsthm,textcomp}
\usepackage{enumerate}
\usepackage{multicol}
\usepackage{tikz}
\usepackage{comment}

\usepackage{geometry}
\geometry{total={210mm,297mm},
left=25mm,right=25mm,%
bindingoffset=0mm, top=20mm,bottom=20mm}


\linespread{1.3}

\newcommand{\linia}{\rule{\linewidth}{0.5pt}}

% custom theorems if needed
\newtheoremstyle{mytheor}
    {1ex}{1ex}{\normalfont}{0pt}{\scshape}{.}{1ex}
    {{\thmname{#1 }}{\thmnumber{#2}}{\thmnote{ (#3)}}}

\theoremstyle{mytheor}
\newtheorem{defi}{Definition}

% my own titles
\makeatletter
\renewcommand{\maketitle}{
\begin{center}
\vspace{2ex}
{\huge \textsc{\@title}}
\vspace{1ex}
\\
\linia\\
\@author \hfill \@date
\vspace{4ex}
\end{center}
}
\makeatother
%%%

% custom footers and headers
\usepackage{fancyhdr,lastpage}
\pagestyle{fancy}
\lhead{}
\chead{}
\rhead{}
\lfoot{}
\cfoot{}
\rfoot{Page \thepage\ /\ \pageref*{LastPage}}
\renewcommand{\headrulewidth}{0pt}
\renewcommand{\footrulewidth}{0pt}
%

%%%----------%%%----------%%%----------%%%----------%%%

\begin{document}

\title{CS 660 Computer Networks - Fall 2014 \\ Final Exam Answer sheet}

\author{Daqing Yi}

\date{}

\maketitle

\begin{enumerate}
\item 
We examined two clean-slate architectures for a new Internet, NDN and XIA. 
Explain how each architecture works and discuss its benefits. 
Which one do you think has the best chance of replacing the current Internet?
Justify your answer.
\paragraph{Answer:}
\begin{itemize}
\item
\textbf{NDN} stands for \emph{Named Data Networking}.
\emph{NDN} is driven by the data consumers.
Two types of packets, \emph{Interest} and \emph{Data}, are forwarded in the communications.
A consumer sends the \emph{interest} packet into the network.
When an \emph{interes} arrives an NDN router, its CS (\emph{Content Store}) will firstly be checked.
A node that has the requested data will return the \emph{data} packet.
A NDN router uses a PIT (\emph{Pending Interest Table}) to store all forwarded but not satisfied \emph{Interests}.
When a \emph{data} packet arrives an NDN router, it will firstly be removed from PIT and then added to CS.
The benefits of \emph{NDN} architecture includes
\begin{itemize}
	\item  End-to-end principle
	\item Routing and forwarding plane separation
	\item Stateful forwarding
	\item Built-in security
	\item Endable user choice and competitin
\end{itemize}
\item
\textbf{XIA} stands for \emph{eXpressive Internet Architecture}.
It aims for providing native supports for different principals so that new functionalities can be incrementally deployed.
Three key ideas 
\begin{itemize}
\item \emph{Flexible principal types} enables the compatibility of new types even before the network provides native support to new functions.
\item \emph{Flexible addressing} provides a gradual network support.
Any new function can be deployed piecewise.
\item \emph{Intrinsically secure identifiers} allows the accuracy and integrity in communication.
\end{itemize}
The benefits of \emph{XIA} architecture includes
\begin{itemize}
	\item content transfer and support for evolution
	\item service migration
	\item client mobility
\end{itemize}
\end{itemize}

Because currently the internet becomes already an international community.
Any overwhelming reimplementation of the internet becomes hard, because it needs profit consensus and cooperation.
It would be ideal to me that there is an standard like \emph{XIA} that supports the coexistence and competition for different architectures.
This can provide the architectures, like \emph{NDN}, the opportunity to start from small deployment but then become popular to win the game.

\item 
We covered papers on domain names and domain parking. 
Explain what these papers tell us about current domain usage. 
What topics are still open for research with respect to Internet domains?
\paragraph{Answer:}
\begin{itemize}
\item Parking service
\item Illicit activities in online advertising
\item Infiltration into malicious infrastructure
\end{itemize}
\begin{comment}
\item 
The paper on SSL, HTTPS, and trust models describes a number of flaws with our system for secure communication. 
Choose three of these you feel are most important. 
Describe the aw and explain why this is a critical area where improvement is needed.
\paragraph{Answer:}
xx
\end{comment}
\item
The paper on reducing web latency describes three methods for improving TCP transfers. 
Briefly explain how each of these works. 
If you were to design a new version of TCP, and deployment was not an issue, which of these would you use and why?
\paragraph{Answer:}
\begin{itemize}
\item \textbf{Proactive} \emph{Proactively recover from loses}
\item \textbf{Reactive} \emph{Recover from them as quickly as possible}
\item \textbf{Corrective} \emph{Reconstruct packets to mask loss}
\end{itemize}
\item
Explain the issue of network neutrality to an audience of a fellow CS major who is unfamiliar with the issue.
Conclude with your advice for how society should solve this problem, or provide justification why nothing should be done from a regulatory standpoint.
\paragraph{Answer:}
\emph{Network neutrality} focuses whether an ISP differentiates against specific end-hosts, protocols or application.
People believe that the internet neutrality preserves the prosperity of the online innovation.
Because this guarantees a hospitable environment to the new websites and internet business and a fair competition between any new service providers with the broadband providers.

As I am from a internet highly-regulated country, I do not believe the government can always guarantee a neutral environment.
However, because the government has more investigation tools. 
They could bring more transparency to the public, so that they can know what is happening in the networks.
At the same time, the government can bring more ISP options to the local residents.
The people and the market can bring it to a balance.

\item
HLP was proposed as a replacement to BGP. 
Explain the major ways that HLP tries to improve on BGP. 
To what extent can these problems can be solved by the simpler solution proposed in ``Putting BGP on the
right path"? 
Which are left unsolved and may require deployment of a new protocol?
\paragraph{Answer:}
\begin{itemize}
\item Scalability
\item Isolation
\item Convergence
\end{itemize}

\end{enumerate}




\end{document}
