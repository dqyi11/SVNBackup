%%% Template originaly created by Karol Kozioł (mail@karol-koziol.net) and modified for ShareLaTeX use

\documentclass[a4paper,11pt]{article}

\usepackage[T1]{fontenc}
\usepackage[utf8]{inputenc}
\usepackage{graphicx}
\usepackage{xcolor}

\usepackage{tgtermes}

\usepackage[
pdftitle={CS660 Final Answer sheet}, 
pdfauthor={Daqing Yi, Brigham Young University},
colorlinks=true,linkcolor=blue,urlcolor=blue,citecolor=blue,bookmarks=true,
bookmarksopenlevel=2]{hyperref}
\usepackage{amsmath,amssymb,amsthm,textcomp}
\usepackage{enumerate}
\usepackage{multicol}
\usepackage{tikz}
\usepackage{comment}

\usepackage{geometry}
\geometry{total={210mm,297mm},
left=25mm,right=25mm,%
bindingoffset=0mm, top=20mm,bottom=20mm}


\linespread{1.3}

\newcommand{\linia}{\rule{\linewidth}{0.5pt}}

% custom theorems if needed
\newtheoremstyle{mytheor}
    {1ex}{1ex}{\normalfont}{0pt}{\scshape}{.}{1ex}
    {{\thmname{#1 }}{\thmnumber{#2}}{\thmnote{ (#3)}}}

\theoremstyle{mytheor}
\newtheorem{defi}{Definition}

% my own titles
\makeatletter
\renewcommand{\maketitle}{
\begin{center}
\vspace{2ex}
{\huge \textsc{\@title}}
\vspace{1ex}
\\
\linia\\
\@author \hfill \@date
\vspace{4ex}
\end{center}
}
\makeatother
%%%

% custom footers and headers
\usepackage{fancyhdr,lastpage}
\pagestyle{fancy}
\lhead{}
\chead{}
\rhead{}
\lfoot{}
\cfoot{}
\rfoot{Page \thepage\ /\ \pageref*{LastPage}}
\renewcommand{\headrulewidth}{0pt}
\renewcommand{\footrulewidth}{0pt}
%

%%%----------%%%----------%%%----------%%%----------%%%

\begin{document}

\title{CS 660 Computer Networks - Fall 2014 \\ Final Exam Answer sheet}

\author{Daqing Yi}

\date{}

\maketitle

\begin{enumerate}
\item 
We examined two clean-slate architectures for a new Internet, NDN and XIA. 
Explain how each architecture works and discuss its benefits. 
Which one do you think has the best chance of replacing the current Internet?
Justify your answer.
\paragraph{Answer:}
\begin{itemize}
\item
\textbf{NDN} stands for \emph{Named Data Networking}.
\emph{NDN} is driven by the data consumers.
Two types of packets, \emph{Interest} and \emph{Data}, are forwarded in the communications.
A consumer sends the \emph{interest} packet into the network.
When an \emph{interes} arrives an NDN router, its CS (\emph{Content Store}) will firstly be checked.
A node that has the requested data will return the \emph{data} packet.
A NDN router uses a PIT (\emph{Pending Interest Table}) to store all forwarded but not satisfied \emph{Interests}.
When a \emph{data} packet arrives an NDN router, it will firstly be removed from PIT and then added to CS.
The benefits of \emph{NDN} architecture includes
\begin{itemize}
	\item the named data routing is robust when there is node failure;
	\item content cache might reduce congestion;
	\item network traffic can be self-regulated;
	\item better inherent security;
	\item it supports user choice and competition as the network evolves.
\end{itemize}
\item
\textbf{XIA} stands for \emph{eXpressive Internet Architecture}.
It aims for providing native supports for different principals so that new functionalities can be incrementally deployed.
Three key ideas 
\begin{itemize}
\item \emph{Flexible principal types} \\
The format of the principal type can support any new type format.
Any new type can be compatible before the network provides native supports to new functions.
\item \emph{Flexible addressing} \\
It provides a gradual network support.
Any new function can be deployed piecewise.
Fallback also helps to specify backup option in case that the new type is not recognized.
\item \emph{Intrinsically secure identifiers} \\
Intrinsic security is required in both source and destination sides.  
\end{itemize}
The benefits of \emph{XIA} architecture includes
\begin{itemize}
	\item The user can specifiy static content with the help of scoped intent;
	\item seamless service migration can be supported by re-binding mechanism;
	\item client mobility is accomplished by re-binding mechanism.
\end{itemize}
\end{itemize}

Because currently the Internet becomes already an international community.
Any overwhelming reimplementation of the Internet becomes hard, because it needs profit consensus and cooperation.
It would be ideal that there is an standard like \emph{XIA} that supports the coexistence and competition for different architectures.
This can provide the architectures, like \emph{NDN}, the opportunity to start from small-scale deployment but then become popular to win the game.

\item 
We covered papers on domain names and domain parking. 
Explain what these papers tell us about current domain usage. 
What topics are still open for research with respect to Internet domains?
\paragraph{Answer:}
Currently there are a large number of domain names that the owners do not have better uses of.
A great share of them are used for domain parking services.
However, some of them are involved in illicit activities, which provide malicious contents.
One of the paper investigates the existing ``dark side'' of domain parking.
Most of them are registered by spammers and appear as typosquatting domain names, because they are more likely to be visited accidentally due to domain typos.
Thus one paper proposes methods to detect typosquatting domain names to protect the users and the other introduces methods to detect spammer registration behaviors to reduce the agility for mounting attacks.

The open topics in Internet domains include
\begin{itemize}
\item the detection of roles of registrars and registries ( including abnormal parking services ),
\item the deep investigations in illicit activities in online advertising,
\item the infiltration into malicious infrastructure,
\item efficient DNS resolution algorithms,
\item and domain generation algorithms.
\end{itemize}
\begin{comment}
\item 
The paper on SSL, HTTPS, and trust models describes a number of flaws with our system for secure communication. 
Choose three of these you feel are most important. 
Describe the aw and explain why this is a critical area where improvement is needed.
\paragraph{Answer:}
xx
\end{comment}

\item
The paper on reducing web latency describes three methods for improving TCP transfers. 
Briefly explain how each of these works. 
If you were to design a new version of TCP, and deployment was not an issue, which of these would you use and why?
\paragraph{Answer:}
Three methods are proposed for reducing web latency.
\begin{itemize}
\item \textbf{Reactive} \emph{Recover from them as quickly as possible} \\
A probe segment is transmitted to trigger duplicate ACKs after a PTO(Probe timeout).
The recovery time is attempted to be shorter than an RTO.
Only changes on sender's side are needed.
\item \textbf{Proactive} \emph{Proactively recover from loses} \\
Copies of each TCP segment is transmitted.
If the receiver gets at least of one copy of the segment, it can discard redudant segments and proceed.
It requires minimum changes on TCP implementation.
\item \textbf{Corrective} \emph{Reconstruct packets to mask loss} \\
The sender transmits extra FEC packets so that the receiver can repair a small number of loses.
Because the measurements indicate that usually only a single packet is lost.
Thus the checksum process could recover many loses.
\end{itemize}

If I could design a new version of TC without worrying about deployment, 
\emph{corrective} would be my preference.
\emph{Proactive} is too aggressive, too many redundant packets are transmitted.
If every user adopts this way, it might raise congestion issue.
The performance of the \emph{reactive} depends on a good RTO.
If PTO is too short, it might cause redundant transmission.
If PTO is too long, the advantage will not be significant.

\emph{Corrective} provides a good balance.
If the assumption from measurement is correct, the recovery should be effective and save the transmission cost.
In many cases, the cost of local processing is cheaper than the cost of transmission.


\item
Explain the issue of network neutrality to an audience of a fellow CS major who is unfamiliar with the issue.
Conclude with your advice for how society should solve this problem, or provide justification why nothing should be done from a regulatory standpoint.
\paragraph{Answer:}
\emph{Network neutrality} focuses whether an ISP differentiates against specific end-hosts, protocols or application.
People believe that the internet neutrality preserves the prosperity of the online innovation.
Because this guarantees a hospitable environment to the new websites and internet business and a fair competition between any new service providers with the broadband providers.

As I am from a internet highly-regulated country, I do not believe the government can always guarantee a neutral environment.
However, because the government has more investigation tools. 
They could bring more transparency to the public, so that they can know what is happening in the networks.
At the same time, the government can bring more ISP options to the local residents.
The people and the market can bring it to a balance.

\item
HLP was proposed as a replacement to BGP. 
Explain the major ways that HLP tries to improve on BGP. 
To what extent can these problems can be solved by the simpler solution proposed in ``Putting BGP on the
right path"? 
Which are left unsolved and may require deployment of a new protocol?
\paragraph{Answer:}
\begin{itemize}
\item BGP fails in scalability test. 
Because the routing state and rate of churn are linearly growing with the network size.
\item When changes happen, BGP usually suffers from route instability, route oscilliation and thus long convergence time.
\item BGP shows poor fault isolation property.
\end{itemize}
By using cost-hiding, HLP could perform explicit information hiding of routing updates.
This makes HLP have better scalability and isolation.
As well, the convergence of HLP has been proved with some assumptions satisfied.

Some BGP policies are not supported by HLP.
Due to path hiding, generic regular expression based policies are not well supported.
Also negation-based expressions on the path vector is not supported. 
Thus some other transit networks cannot be avoided in the routing.

\end{enumerate}




\end{document}
