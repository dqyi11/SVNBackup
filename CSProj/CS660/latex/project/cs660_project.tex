%File: formatting-instruction.tex
\documentclass[letterpaper]{article}
\usepackage{aaai}
\usepackage{times}
\usepackage{helvet}
\usepackage{courier}
\usepackage{hyperref}
\usepackage{comment}
\usepackage{graphicx}
\usepackage{natbib}

\usepackage{url}
\usepackage{hyperref}

\usepackage{amsmath,amsfonts,amsthm} % Math packages
\usepackage{algorithm}
%\usepackage{algorithmicx}
\usepackage[noend]{algpseudocode}

\frenchspacing
\setlength{\pdfpagewidth}{8.5in}
\setlength{\pdfpageheight}{11in}
\pdfinfo{
/Title (Mobile Indoor Localization)
/Author (Daqing Yi)}
\setcounter{secnumdepth}{0}  
\begin{document}
% The file aaai.sty is the style file for AAAI Press 
% proceedings, working notes, and technical reports.
%
\title{Community discovery in social networks using Latent Dirichlet Allocation}
%\subtitle{CS 660 Computer Networks}

\author{ 
	Joseph Walker 
	\\ \texttt{joseph.k.walker@gmail.com}
	\And Seung-Hee Yang 
	\\ \texttt{yangseung0105@gmail.com}
	\And Daqing Yi 
	\\ \texttt{dqyi11@gmail.com}
}
\maketitle

\section{Introduction}

\section{Related work}

\section{Graph generative model}

\emph{Social interaction profile} characterizes each actor.
For actor $ v_{i} $, it is defined as a set of neighbor $ \omega_{i,j} $ and the corresponding weight $ \mbox{SIW}(v_{i}, \omega_{i,j}) $.
Thus,
\begin{equation}
\mbox{SIP}(v_{i}) = \{ (\omega_{i,1}, \mbox{SIW}(v_{i}, \omega_{i,1})) , \cdots , (\omega_{i,m_{i}}, \mbox{SIW}(v_{i}, \omega_{i,m_{i}}) \},
\end{equation}
$ m_{i} $ is the size of $ v_{i} $'s social interaction profile and $ \mbox{SIW}() $ defines the strength of such interaction.
A social network contains a set of communities $ \iota( \iota_{1} , \iota_{2} , \cdots , \iota_{k} ) $.

Here is a list of notations.
\begin{itemize}
	\item $ M $ : number of social actors (social interaction profiles) in the social network
	\item $ K $ : number of communities (mixture components)
	\item $ N_{i} $ : number of social interactions in a social interaction profile $ \mbox{sip}_{i} $
	\item $ \vec{\alpha} $ : Dirichlet prior hyperparameter
	\item $ \vec{\beta} $ : Dirichlet prior hyperparameter
	\item $ \iota $ : hidden community variable, $ \iota_{i,j} $ means the community for the $ j $th social interaction in $ \mbox{sip}_{i} $
	\item $ \vec{\theta} $ : $ p(\iota \mid \mbox{sip}_{j}) $ is the community mixture proportion for $ \mbox{sip}_{j} $ 
	\item $ \vec{\Phi}_{k} $ : $ p(\omega \mid \iota_{k} ) $ is the mixture component of community $ k $
	\item $ \omega $ : social interaction variable, $ $ means the $ j $th social interaction in $ \mbox{sip}_{i} $
\end{itemize}

The generative process for an agent $ \omega_{i} $'s social interaction profile $ \mbox{sip}_{i} $ in a social network.
\begin{itemize}
	\item Sample mixture components $ \vec{\phi}_{k} \sim \mbox{Dirichlet}(\vec{\beta}) $ for $ k \in [1, K] $
	\item Choose $ \vec{\theta}_{i} \sim \mbox{Dirichlet}(\vec{\alpha}) $ 
	\item Choose $ N_{i}  \sim \mbox{Poisson}(\xi) $
	\item For each of the $ N_{i} $ social interactions $ \omega_{i,j} $:
	\begin{itemize}
		\item Choose a community $ \iota_{i,j} \sim \mbox{Multinomial}( \vec{\theta}_{i} ) $
		\item Choose a social interaction $ \omega_{i,j} \sim \mbox{Multinomial}( \vec{\theta}_{\iota_{i,j}} ) $	
	\end{itemize}
\end{itemize}

The probability that the $ j $th social interaction element $ \omega_{i,j} $ in the social actor $ \omega_{i} $'s social interaction profile $ \mbox{sip}_{i} $ instantiates a particular neighboring agent $ \omega_{m} $ is
\begin{equation}
p( \omega_{i,j} = \omega_{m} \mid \vec{\theta}_{i} , \underline{\mathbf{\Phi}} ) = \sum_{k=1}^{K} p( \omega_{i,j} = \omega_{m} \mid \vec{\Phi}_{k} ) p( \iota_{i,j} = k \mid \vec{\theta}_{i} ),
\end{equation}
where $ \vec{\theta}_{i} $ is the mixing proportion variable for $ \mbox{sip}_{i} $ and $ \vec{\Phi}_{k} $ is the parameter set for the $ k $th community component distribution.
The joint distribution of all known and hidden variables is
\begin{equation}
p( \vec{\omega}_{i} , \vec{\iota}_{i}, \vec{\theta}_{i} , \underline{\mathbf{\Phi}}  \mid \vec{\alpha} , \vec{\beta} ) = \prod_{j=1}^{N_{i}} 
p( \omega_{i,j} \mid \vec{\phi}_{\iota_{i,j}} ) P( \iota_{i,j} \mid \vec{\theta}_{i} )
p( \vec{\theta}_{i} \mid \vec{\alpha} ) p( \underline{\mathbf{\Phi}} \mid \vec{\beta} )
\end{equation}

\section{Latent Dirichlet Allocation}

The desired distribution is the posterior given evidence
\begin{equation}
p( \iota \mid \omega ) = \frac{ p( \omega , \iota ) }{ \sum_{ \iota } p( \omega , \iota ) }.
\end{equation}
Using a Gibbs sampling, the joint distribution is
\begin{equation}
\begin{aligned}
p( \vec{\omega}, \vec{\iota} \mid \vec{\alpha}, \vec{\beta} )  & = p( \vec{\omega} \mid \vec{\iota}, \vec{\beta} ) p( \vec{\iota} \mid \vec{\alpha} ) \\
& = \prod_{\iota = 1}^{K} \frac{\Delta(\vec{n}_{\iota}+\vec{\beta})}{\Delta{\vec{\beta}}}
\prod_{m=1}^{M} \frac{\Delta(\vec{n}_{m}+\vec{\alpha})}{\Delta{\vec{\alpha}}}
\end{aligned}
\end{equation}

The update equation for the hidden variables can be derived
\begin{equation}
p( \iota_{i} = j \mid \vec{\iota}_{\neg i}, \vec{\omega}) \propto 
\frac{ n^{\omega_{i}}_{\neg i, j}+\beta }{ n^{\cdot}_{\neg i, j}+W\beta } \ast \frac{ n^{\mbox{sip}_{i}}_{\neg i}+\alpha }{ n^{\mbox{sip}_{i}}_{\neg i}+T\alpha }
\end{equation}

The update formula for $ $ and $ $ are
\begin{equation}
\phi_{k,\omega} = \frac{n^{\omega}_{k}+\beta}{\sum_{v=1}^{V}n^{v}_{k}+W\beta}
\end{equation}
and
\begin{equation}
\theta_{m,k} = \frac{n^{k}_{m}+\alpha}{\sum_{K}^{\iota=1}+T\alpha}.
\end{equation}

The pseudo code is given as following.
\begin{algorithm}
	\begin{algorithmic}[1]
		\For{\textbf{each} social interaction profile $ \mbox{sip}_{i} \in [1,M] $ }
		\For{\textbf{each} social interaction $ \omega_{i,j} \in [1, N_{i}] $ }
		\State sample community index $ \iota_{i,j} \sim \mbox{Mult}(\frac{1}{K}) $
		\State update counters: $ n^{\iota_{i,j}}_{i} + 1 $, $ n_{i} + 1 $, $ n^{\iota_{i,j}}_{\omega_{i,j}} +1 $ and $ n_{\iota_{i,j}} +1 $
		\EndFor
		\EndFor
		
		\While{not finished}
		\For{\textbf{each} SIP $ \mbox{sip}_{i} $}
		\For{\textbf{each} $ \omega_{i,j} \in [1, N_{i}] $}
		\State decrement counters and sums:  $ n^{\iota_{i,j}}_{i} - 1 $, $ n_{i} - 1 $, $ n^{\iota_{i,j}}_{\omega_{i,j}} - 1 $ and $ n_{\iota_{i,j}} - 1 $ 
		\State resample $ \omega_{i,j} $ according to  $ p(\iota_{i}=j \mid \vec{\iota}_{\neg i}, \vec{\omega} ) $
		\State update counter accordingly
		\EndFor
		\EndFor
		\If{converged and L iterations}
		\State update parameters $ \phi $ and $ \theta $
		\EndIf
		\EndWhile
	\end{algorithmic}
	\caption{Gibbs sampling process}
\end{algorithm}

\section{Social Network Structure}


\section{Analysis}

\section{Summary}

\bibliographystyle{abbrv}
\bibliography{reference}

\end{document}
