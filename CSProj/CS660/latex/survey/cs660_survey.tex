% THIS IS SIGPROC-SP.TEX - VERSION 3.1
% WORKS WITH V3.2SP OF ACM_PROC_ARTICLE-SP.CLS
% APRIL 2009
%
% It is an example file showing how to use the 'acm_proc_article-sp.cls' V3.2SP
% LaTeX2e document class file for Conference Proceedings submissions.
% ----------------------------------------------------------------------------------------------------------------
% This .tex file (and associated .cls V3.2SP) *DOES NOT* produce:
%       1) The Permission Statement
%       2) The Conference (location) Info information
%       3) The Copyright Line with ACM data
%       4) Page numbering
% ---------------------------------------------------------------------------------------------------------------
% It is an example which *does* use the .bib file (from which the .bbl file
% is produced).
% REMEMBER HOWEVER: After having produced the .bbl file,
% and prior to final submission,
% you need to 'insert'  your .bbl file into your source .tex file so as to provide
% ONE 'self-contained' source file.
%
% Questions regarding SIGS should be sent to
% Adrienne Griscti ---> griscti@acm.org
%
% Questions/suggestions regarding the guidelines, .tex and .cls files, etc. to
% Gerald Murray ---> murray@hq.acm.org
%
% For tracking purposes - this is V3.1SP - APRIL 2009

\documentclass{acm_proc_article-sp}
\usepackage{hyperref}

\begin{document}

\title{Mobile Indoor Localization}
\subtitle{CS 660 Computer Networks}

\numberofauthors{1} 
\author{
\alignauthor
Daqing Yi\titlenote{Brigham Young University}\\
       \email{dqyi11@gmail.com}
}

\date{}


\maketitle
%\begin{abstract}
%\end{abstract}

% A category with the (minimum) three required fields
%\category{H.4}{Information Systems Applications}{Miscellaneous}
%A category including the fourth, optional field follows...
%\category{D.2.8}{Software Engineering}{Metrics}[complexity measures, performance measures]

%\terms{Theory}

%\keywords{ACM proceedings, \LaTeX, text tagging} % NOT required for Proceedings

\section{Introduction}

An \emph{Indoor positioning system} locates the objects or people in an indoor environment.
The information used for localization includes radio waves, magnetic fields, acoustic signals or other sensory information from mobile devices.

Indoor localization enables indoor navigation in shopping malls, indoor location-based advertisements, friends and family member tracking and etc.

GPS
\cite{Nirjon:2014:CIL:2594368.2594378} 
WiFi
\cite{Sen:2013:AMR:2462456.2464463}
Smartphone
\cite{Liu:2013:GEF:2462456.2464450}

\cite{Rai:2012:ZZC:2348543.2348580}

\cite{Wang:2012:NNW:2307636.2307655}
Graphical model
\cite{Nandakumar:2012:CLD:2348543.2348579}

\section{GPS assisted localization}
\label{sec:gps_local}

GPS has been applied widely into the 
Due to low signal strength and multipath effect, usually the GPS receiver does not work well in an indoor building.
The indoor signal strength can be reduced 10 to 100 times of that in outdoor.

\cite{Nirjon:2014:CIL:2594368.2594378} proposes a hardware-software approach \emph{COIN-GPS}, which is a highly-sensitivity cloud-offloaded instant GPS.
It consists of three components,
\begin{itemize}
\item a directional antenna,
\item a robust acquisition algorithm and
\item a multi-directional location estimation algorithm.
\end{itemize}

\subsection{Robust acquisition}
Because of the signal noise, the peak detection in common acquisition becomes difficult.
The use of a directional antenna helps reduce the noise.
 
\subsection{Multi-direction location estimation}


 
\section{Wifi assisted localization}

Another popular 
 

\bibliographystyle{abbrv}
\bibliography{reference}

%\balancecolumns 

\end{document}
