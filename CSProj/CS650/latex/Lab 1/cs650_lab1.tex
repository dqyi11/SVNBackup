\documentclass[paper=a4, fontsize=11pt]{scrartcl}
\usepackage[T1]{fontenc}
\usepackage{fourier}

\usepackage[english]{babel}															% English language/hyphenation
\usepackage[protrusion=true,expansion=true]{microtype}	
\usepackage{amsmath,amsfonts,amsthm} % Math packages
\usepackage[pdftex]{graphicx}	
\usepackage{url}
\usepackage{hyperref}


%%% Custom sectioning
\usepackage{sectsty}
\allsectionsfont{\centering \normalfont\scshape}
\usepackage{subfigure}


%%% Custom headers/footers (fancyhdr package)
\usepackage{fancyhdr}
\pagestyle{fancyplain}
\fancyhead{}											% No page header
\fancyfoot[L]{}											% Empty 
\fancyfoot[C]{}											% Empty
\fancyfoot[R]{\thepage}									% Pagenumbering
\renewcommand{\headrulewidth}{0pt}			% Remove header underlines
\renewcommand{\footrulewidth}{0pt}				% Remove footer underlines
\setlength{\headheight}{13.6pt}


%%% Equation and float numbering
\numberwithin{equation}{section}		% Equationnumbering: section.eq#
\numberwithin{figure}{section}			% Figurenumbering: section.fig#
\numberwithin{table}{section}				% Tablenumbering: section.tab#


%%% Maketitle metadata
\newcommand{\horrule}[1]{\rule{\linewidth}{#1}} 	% Horizontal rule

\title{
		%\vspace{-1in} 	
		\usefont{OT1}{bch}{b}{n}
		\normalfont \normalsize \textsc{CS650 - Computer Vision} \\ [25pt]
		\horrule{0.5pt} \\[0.4cm]
		\huge Programming Lab 1 - Image Processing \\
		\horrule{2pt} \\[0.5cm]
}
\author{
		\normalfont 								\normalsize
        Daqing Yi\\[-3pt]		\normalsize
        \today
}
\date{}


%%% Begin document
\begin{document}
\maketitle

Prepare a brief (but specific and well-articulated) write-up about your methods, observations, and conclusions.
Include some images to illustrate your results (you may need to crop and zoom so we can see detail - we do not necessarily need entire images or all results, but enough to be representative of what you have done and demonstrate what you are writing about).

\begin{itemize}
\item What do you conclude about binarization? 
\item What do you conclude about the use of morphological open and close operations?
\end{itemize}

There are no specific right or wrong answers--do your best to think about the methods, their strengths, and their weaknesses.

\section{Binarization}

Otsu's method finds a threshold that separates the pixels into foreground class and background class.
The objective is defined as minimizing the intra-class variance $ \omega_{f} (t) \sigma_{f}^{2} (t) + \omega_{b} (t) \sigma_{b}^{2} (t) $, which equals to maximizing inter-class variance $ \omega_{f} (t) \omega_{b} (t) [ \mu_{f} (t) - \mu_{b} (t) ]^{2} $.
$ t \in [1, N] $ is the intensity level.
$ \omega_{f} (t) $ and $ \omega_{b} (t) $ are the weights of two classes, which are calculated from histogram $ \sum_{i=1}^{t} p(i) $ and $ \sum_{i=t+1}^{N} p(i) $.
$ \mu_{f} (t) $ and $ \mu_{b} (t) $ are the means of two classes, which are computed from $ \sum_{i=1}^{t} i * p(i) $ and $ \sum_{i=t+1}^{N} i * p(i) $.

\begin{figure}
\centering
\includegraphics[width=0.5\linewidth]{./figure/interclass_variances}
\caption{Finding the maximum interclass variance}
\label{fig:interclass_variances}
\end{figure}

Otsu's method uses an exhaustive search to find the optimal solution, like in Figure \ref{fig:interclass_variances}. 

\subsection{Implementation}

The functions are written in \emph{binarization.py}.

\begin{itemize}
\item \textbf{ otsu(hist, total) } reads the histogram and total number of the pixels, then returns the threshold.
\item \textbf{ binarize(img\_data, threshold, on\_value=1) } converts the image data array into binary data array by threshold.
On\_value indicates the number stored for representing ON pixel.
\end{itemize}

\subsection{Results}

\subsubsection{0397.pgm}

The results of testing \emph{0397.pgm} are illustrated in Figure \ref{fig:binary:01}.

\begin{figure}[h]
\centering
\subfigure[Origin]{
\includegraphics[width=.45\textwidth]{./figure/0397_pgm_origin}
\label{fig:binary:01:origin} }
\subfigure[Binary]{
\includegraphics[width=.45\textwidth]{./figure/0397_pgm_binary}
\label{fig:binary:01:binary} }
\caption{Binarization of 0397.pgm using Otsu's method.}\label{fig:binary:01}
\end{figure}

\subsubsection{Other test files}

In order to check the correctness of the implementation, I compared the results from my implementation with the results from modules provided by open cv.
They are given in Table 1.1. %\ref{tab:title}.
When the thresholds are the same, the generated binary images should be consistent as well.
\begin{table}
\label{tab:binary_comp}
\caption {Comparison on the results of \textbf{threshold} obtained from different implementations of Otsu's method.}
\begin{center}
\begin{tabular}{ | l | l | l | }
\hline
Test file & Threshold (My code) & Threshold (Open CV)  \\ \hline
0397.pgm                                          & 134 & 134 \\ \hline
020206\_131612\_bp001\_folio\_094\_k639\_1837.ppm & 182 & 182 \\ \hline
Declaration\_Pg1of1\_AC\_crop.pgm                 & 186 & 186 \\ \hline
Scan\_half\_crop\_norm\_009\_small.pgm            & 134 & 134 \\ \hline
seq-4\_small.pgm                                  & 152 & 152 \\ \hline
\end{tabular}
\end{center}
\end{table}


\section{Morphology}

The morphology filter uses a mask to count the number of ON pixels of the neighborhood of each pixel.
For a binary image that is represented by 0 and 1, either pixel of 0 or pixel of 1 can be defined as ON pixel. 
In this lab, I choose pixel of 0 (BLACK pixel) as ON pixel.

The \textbf{dilate} operation requires all the neighbor pixels in the mask are all ON so that the pixel can be set as ON, 
while the \textbf{erode} operation requires only one of the neighbor pixles.
The \texbtf{open} and \textbf{close} operations are combinations of the \textbf{dilate} and \textbf{erode} operations.
The \textbf{open} operation means doing \textbf{dilate} firstly and then \textbf{erode}.
THe \textbf{close} operation follows a reversed sequence.

\subsection{Implementation}

The functions are written in \emph{morphology.py}.

\begin{itemize}
\item \textbf{ countNum( img\_data, maskSize ) } counts the number of ON pixels in the neighboring of each pixels. 
Because the mask cannot be completely applied in the corners, an array of mask data size for each pixel is returned as well.
\item \textbf{ dataThreshold( data, threshold, ratio=1.0 ) } determines whether the number of ON pixels can surpass the threshold defined.
\end{itemize}

There is also a class \emph{MorphologicalFiltering}.
It stores the \emph{maskSizeData} used for operations \textbf{dilate}, \textbf{erode}, \textbf{open} and \textbf{close}.

\subsection{Results}

\subsubsection{Letter J}

\begin{figure}
\centering
\subfigure[Binary]{
\includegraphics[width=.18\textwidth]{./figure/J_binary}
\label{fig:j_binary} }
\subfigure[Dilate]{
\includegraphics[width=.18\textwidth]{./figure/J_dilate}
\label{fig:j_dilate} }
\subfigure[Erode]{
\includegraphics[width=.18\textwidth]{./figure/J_erode}
\label{fig:j_erode} }
\subfigure[Open]{
\includegraphics[width=.18\textwidth]{./figure/J_open}
\label{fig:j_open} }
\subfigure[Close]{
\includegraphics[width=.18\textwidth]{./figure/J_close}
\label{fig:j_close} }
\caption{Morphology operations of Letter J.}\label{fig:j_letter}
\end{figure}

\subsubsection{0397.pgm}

\begin{figure}[h]
\centering
\subfigure[Origin]{
\includegraphics[width=.45\textwidth]{./figure/0397_pgm_open}
\label{fig:morph:01:open} }
\subfigure[Binary]{
\includegraphics[width=.45\textwidth]{./figure/0397_pgm_close}
\label{fig:morph:01:close} }
\caption{Open and close operations of 0397.pgm.}\label{fig:morph:01}
\end{figure}

\subsubsection{Other test files}

In order to verify the correctness of the generated results, I use $ mean( abs( img - img\_ref ) ) $ to compare the result of my implementation with the result of open cv.
$ img $ is the data array from my implementation and $ img\_ref $ is the data array from open cv.
Because open cv use pixel of 1 as ON pixel, which is the reverse of my way.
In comparison, I compare my ``dilate'' to with the ``erode'' in open cv, and my ``erode'' with ``dilate'' in open cv.
Similarly, because ``open'' and ``close'' are reversed operation, I compare them in the same way. 

\begin{table}
\label{tab:morphology_comp}
\caption {Comparison on the results of \textbf{threshold} obtained from different implementations of Morphological filter.}
\begin{center}
\begin{tabular}{ | l | l | l | l | l | }
\hline
Test file & Dilate & Erode & Open & Close  \\ \hline
Letter J                                          & 0.0            & 0.0           & 0.0            & 0.0            \\ \hline
0397.pgm                                          & 0.0            & 0.0           & 0.0            & 0.0            \\ \hline
020206\_131612\_bp001\_folio\_094\_k639\_1837.ppm & 0.694292911729 & 1.11458020782 & 0.789870370183 & 0.623459113607 \\ \hline
Declaration\_Pg1of1\_AC\_crop.pgm                 & 0.0            & 0.0           & 0.0            & 0.0 \\ \hline
Scan\_half\_crop\_norm\_009\_small.pgm            & 0.0 & 0.0 & 0.0 & 0.0 \\ \hline
seq-4\_small.pgm                                  & 0.0 & 0.0 & 0.0 & 0.0 \\ \hline
\end{tabular}
\end{center}
\end{table}

\end{document}