\documentclass[paper=a4, fontsize=11pt]{scrartcl}
\usepackage[T1]{fontenc}
\usepackage{fourier}

\usepackage[english]{babel}															% English language/hyphenation
\usepackage[protrusion=true,expansion=true]{microtype}	
\usepackage{amsmath,amsfonts,amsthm} % Math packages
\usepackage[pdftex]{graphicx}	
\usepackage{url}
\usepackage{hyperref}


%%% Custom sectioning
\usepackage{sectsty}
\allsectionsfont{\centering \normalfont\scshape}
\usepackage{subfigure}

\usepackage{comment}


%%% Custom headers/footers (fancyhdr package)
\usepackage{fancyhdr}
\pagestyle{fancyplain}
\fancyhead{}											% No page header
\fancyfoot[L]{}											% Empty 
\fancyfoot[C]{}											% Empty
\fancyfoot[R]{\thepage}									% Pagenumbering
\renewcommand{\headrulewidth}{0pt}			% Remove header underlines
\renewcommand{\footrulewidth}{0pt}				% Remove footer underlines
\setlength{\headheight}{13.6pt}


%%% Equation and float numbering
\numberwithin{equation}{section}		% Equationnumbering: section.eq#
\numberwithin{figure}{section}			% Figurenumbering: section.fig#
%\numberwithin{table}{section}				% Tablenumbering: section.tab#


%%% Maketitle metadata
\newcommand{\horrule}[1]{\rule{\linewidth}{#1}} 	% Horizontal rule

\title{
		%\vspace{-1in} 	
		\usefont{OT1}{bch}{b}{n}
		\normalfont \normalsize \textsc{CS650 - Computer Vision} \\ [25pt]
		\horrule{0.5pt} \\[0.4cm]
		\huge Programming Lab 2 \\ Edge Detection / Hough Transform \\
		\horrule{2pt} \\[0.5cm]
}
\author{
		\normalfont 								\normalsize
        Daqing Yi\\[-3pt]		\normalsize
        \today
}
\date{}


%%% Begin document
\begin{document}
\maketitle

\begin{comment}
Prepare a brief writeup that includes each part and submit it as a PDF through Canvas.
Your writeup should document your findings for Part 1 but otherwise focus on your methods and results for Part 2.

Note: many of the images for Part 1 will contain negative numbers or numbers larger than 255.
Make sure you appropriately scale the output images to display all of the information.
(Hint: if there are negative values, try mapping 0 to 128 with positive values mapped to the range (128,255] and negative values to the range [0,128).]
\end{comment}

This lab consists of two parts, edge detection and hough transform.
The implementations are written in Python.
PIL is used for reading image files into data arrays.
Numpy is used for array operations.
Matplotlib and imshow (from openCV) is used for visualizing data.

\section{Edge detection}

Edge detection aims at finding the edges inside the images by the clues.
The representation is usually a set of points.
The implementation is in \emph{EdgeDetection.py}.

\subsection{First and Second Order Derivatives of an Image}

\begin{figure}[h]
\centering
\subfigure[Origin]{
\includegraphics[width=.23\textwidth]{./figure/2D_White_Box_origin}
\label{fig:edge:01:origin} }
\subfigure[Gradient Magnitude]{
\includegraphics[width=.23\textwidth]{./figure/2D_White_Box_gradient_magnitude} 
\label{fig:edge:01:gradient_magnitude} }
\subfigure[Gradient Orientation]{
\includegraphics[width=.23\textwidth]{./figure/2D_White_Box_gradient_orientation}
\label{fig:edge:01:gradient_orientation} }
\subfigure[Laplacian]{
\includegraphics[width=.23\textwidth]{./figure/2D_White_Box_laplacian} 
\label{fig:edge:01:laplacian} }
\caption{Derivatives of \emph{2D\_White\_Box.pgm} by 1st order and 2nd order.}
\label{fig:edge:01}
\end{figure}

The clues for finding the edges are obtained from differential geometry.
Figure \ref{fig:edge:01} and Figure \ref{fig:edge:02} give the examples on 1st order and 2nd order derivatives.
Figure \ref{fig:edge:01:gradient_magnitude} and Figure \ref{fig:edge:02:gradient_magnitude} illustrate the gradient magnitude, which shows the strength of intensity variation of each pixel.
Figure \ref{fig:edge:01:gradient_orientation} and Figure \ref{fig:edge:02:gradient_orientation} provide the gradient orientation, which provides the direction of intensity variation of each pixel.
Figure \ref{fig:edge:01:laplacian} and Figure \ref{fig:edge:02:laplacian} offer the 2nd derivatives, which can be used to identify the local maximum or minimum.

\begin{figure}[h]
\centering
\subfigure[Origin]{
\includegraphics[width=.23\textwidth]{./figure/blocks_origin}
\label{fig:edge:02:origin} }
\subfigure[Gradient Magnitude]{
\includegraphics[width=.23\textwidth]{./figure/blocks_gradient_magnitude} 
\label{fig:edge:02:gradient_magnitude} } 
\subfigure[Gradient Orientation]{
\includegraphics[width=.23\textwidth]{./figure/blocks_gradient_orientation}
\label{fig:edge:02:gradient_orientation} }
\subfigure[Laplacian]{
\includegraphics[width=.23\textwidth]{./figure/blocks_laplacian} 
\label{fig:edge:02:laplacian} }
\caption{Derivatives of \emph{blocks.pgm} by 1st order and 2nd order.}
\label{fig:edge:02}
\end{figure}

\subsection{edge detection}

With the clues given, edge detection methods can be applied.
Two types of methods are implemented in \emph{EdegDetection.py}, which are \textbf{canny edge detector} and \textbf{zero-crossing edge detector}.

\subsubsection{Canny edge detector}

Canny edge detector is implemented in \emph{canny} in \emph{EdegDetection.py}.
Methods \emph{nonmaximalSuppresion()} and \emph{gaussianFilter()} are used to preprocess before applying the threshold binarization.
\emph{gaussianFilter()} can smooth the distribution.
\emph{nonmaximalSuppresion()} can thinner the cluster of the points by comparing the neighborhood.
It relies on the gradient orientation.
A \emph{hysteresisThreshold()} is applied for binarization based on the gradient magnitude.
A pixel above the low threshold but below the high threshold needs checking its connected pixels in determining whether it is an edge point.


\begin{figure}[h]
\centering
\subfigure[hi\_threshod=40, lo\_threshod=20]{
\includegraphics[width=.235\textwidth]{./figure/blocks_canny_20_40}
\label{fig:edge:02:canny:1} }
\subfigure[hi\_threshod=60, lo\_threshod=30]{
\includegraphics[width=.235\textwidth]{./figure/blocks_canny_30_60} 
\label{fig:edge:02:canny:2} } 
\subfigure[hi\_threshod=80, lo\_threshod=40]{
\includegraphics[width=.235\textwidth]{./figure/blocks_canny_40_80}
\label{fig:edge:02:canny:3} }
\subfigure[hi\_threshod=60, lo\_threshod=30, with Gaussian filter]{
\includegraphics[width=.235\textwidth]{./figure/blocks_canny_gauss_30_60} 
\label{fig:edge:02:canny:gauss} }
\caption{Canny edge detector on \emph{blocks.pgm} using different parameters.}
\label{fig:edge:02:canny}
\end{figure}

Different parameters are tried. 
The difference on the edges found are illustrated in Figure {fig:edge:02:canny}.
Lowering the low threshold tends to connect the disconnected edges.
Increasing the high threshold decreases the edge details that can be found.

\subsection{Laplacian zero-crossing edge detector}

A variant of canny edge detector uses the zero-crossing of the 2nd derivatives.
Because it is linked to Marr–Hildreth (zero crossing of the Laplacian) edge detector \footnote{\url{http://en.wikipedia.org/wiki/Canny_edge_detector}}, the function name is \emph{mh()} in \emph{EdgeDetection.py}.

\begin{figure}[h]
\centering
\subfigure[Threshold=40]{
\includegraphics[width=.235\textwidth]{./figure/blocks_mh_40}
\label{fig:edge:02:mh:1} }
\subfigure[Threshold=60]{
\includegraphics[width=.235\textwidth]{./figure/blocks_mh_60} 
\label{fig:edge:02:mh:2} } 
\subfigure[Threshold=80]{
\includegraphics[width=.235\textwidth]{./figure/blocks_mh_80}
\label{fig:edge:02:mh:3} }
\subfigure[Threshold=40, with Gaussian filter]{
\includegraphics[width=.235\textwidth]{./figure/blocks_mh_gauss_40} 
\label{fig:edge:02:mh:gauss} }
\caption{Laplacian zero-crossing edge detector on \emph{blocks.pgm} using different parameters.}
\label{fig:edge:02:mh}
\end{figure}

This method checks whether there exists a zero crossing in a pixel along the direction of the gradient orientation.
If the pixel is a zero-crossing point, it means that it is likely to be a local maximum or minimum. 
A threshold is also applied here to filter some levels of noise.
Increasing the threshold removes the noise and small details, which is illustrated in Figure \ref{fig:edge:02:mh}.

\subsection{Discussion}



When a smooth filter is needed?
Figure \ref{fig:edge:02:mh:gauss} gives an example.

\section{Hough transformation}

Hough transform is used for finding the features from detected edges.
It tries to find the points in the parameter space that maximize the likelihood of the distribution of the edge points.
From an edge point, all possible solutions in the parameter space could be found.
By accumulating all the possible parameters, we can have a parameter that connects to more edge points with a relatively higher value.
The local maximum then are chosen as the features found in the hough space.

\subsection{Hough circle transformation}

This lab aims at finding the circles in the pictures. 
Only three radii are tried, which are $ 16 $, $ 32 $ and $ 48 $ respectively.
Thus only two parameters, which are the x-coordinate and the y-coordinate of the centers of the circles, are left to explore.
The implementation is in \emph{HoughTransform.py}.
\emph{houghCircle()} provides a 2-D array for the accumulations of two parameters, which can be visualized into a 2-D image for the hough space.

\emph{simple\_circle.pgm} provides a simple case to verify the correctness of the function. 
Figure \ref{fig:hough:simple_circle} gives the process of finding the simple circle using a radius of 32.
The center found by running the code is $ (124, 127) $.

\begin{figure}[h]
\centering
\subfigure[Edges]{
\includegraphics[width=.31\textwidth]{./figure/simplecircles_ppm_canny}
\label{fig:hough:simple_circle:edge} }
\subfigure[Hough space]{
\includegraphics[width=.31\textwidth]{./figure/simplecircles_ppm_hough} 
\label{fig:hough:simple_circle:hough} } 
\subfigure[Circles found]{
\includegraphics[width=.31\textwidth]{./figure/simplecircles_ppm_circles}
\label{fig:hough:simple_circle:circles} }
\caption{Detecting circles (radius=32) on \emph{circles.ppm}.}
\label{fig:hough:simple_circle}
\end{figure}

\subsection{Finding different circle features}

\emph{circles.ppm} is used to find the features in an image mixed with a lot of different features and additive noise.
Because the center might locate outside of the image.
If the image size is $ (img_width, img_height) $, a range of $ [-radius, img_width + radius] $ will be scanned for the x coordinate and a range of $ [-radius, img_height + radius ] $ will be scanned for the y coordinate.

\begin{itemize}
\item $ radius = 16 $ : The centers of the circles found are (89, 115), (-7, 122), (28, 144), (28, 149), (25, 150), (90, 196), (52, 197) .
\item $ radius = 32 $ : The centers of the circles found are (46, 32), (57, 34), (150, 34), (220, 37), (158, 67), (79, 102), (89, 115), (173, 189), (186, 208) .
\item $ radius = 48 $ : The centers of the circles found are  (0, 103), 
(0, 108), (89, 115), (71, 206) .
\end{itemize}

\begin{figure}[h]
\centering
\subfigure[Edges]{
\includegraphics[width=.31\textwidth]{./figure/circles_ppm_canny_16}
\label{fig:hough:circle:16:edge} }
\subfigure[Hough space]{
\includegraphics[width=.31\textwidth]{./figure/circles_ppm_hough_16} 
\label{fig:hough:circle:16:hough} } 
\subfigure[Circles found]{
\includegraphics[width=.31\textwidth]{./figure/circles_ppm_circles_16}
\label{fig:hough:circle:16:circles} }
\caption{Detecting circles (radius=16) on \emph{circles.ppm}.}
\label{fig:hough:circle:16}
\end{figure}


\begin{figure}[h]
\centering
\subfigure[Edges]{
\includegraphics[width=.31\textwidth]{./figure/circles_ppm_canny_32}
\label{fig:hough:circle:32:edge} }
\subfigure[Hough space]{
\includegraphics[width=.31\textwidth]{./figure/circles_ppm_hough_32} 
\label{fig:hough:circle:32:hough} } 
\subfigure[Circles found]{
\includegraphics[width=.31\textwidth]{./figure/circles_ppm_circles_32}
\label{fig:hough:circle:32:circles} }
\caption{Detecting circles (radius=32) on \emph{circles.ppm}.}
\label{fig:hough:circle:32}
\end{figure}

\begin{figure}[h]
\centering
\subfigure[Edges]{
\includegraphics[width=.31\textwidth]{./figure/circles_ppm_canny_48}
\label{fig:hough:circle:48:edge} }
\subfigure[Hough space]{
\includegraphics[width=.31\textwidth]{./figure/circles_ppm_hough_48} 
\label{fig:hough:circle:48:hough} } 
\subfigure[Circles found]{
\includegraphics[width=.31\textwidth]{./figure/circles_ppm_circles_48}
\label{fig:hough:circle:48:circles} }
\caption{Detecting circles (radius=48) on \emph{circles.ppm}.}
\label{fig:hough:circle:48}
\end{figure}

Figure \ref{fig:hough:circle:16}, Figure \ref{fig:hough:circle:32} and Figure \ref{fig:hough:circle:48} provide the visualized processes and results of finding circles with different radii.
By human's visual experience, it is obvious that whether the circles are easy to tell depends on the edges detected.
In order to enhance the performance of edge detection, different parameters are applied to edge detection so that different patterns of edges could be detected.

\subsection{Apply different threshold for the centers outside of the image}

Because the center outside the image tends to make less edge points in the image, a different threshold is applied in finding the local maximum in the region outside of the image.

$ radius = 32 $ : The centers of the circles found are (-7, -8), (200, -4), (150, 34), (220, 37), (158, 67), (199, 116), (208, 178), (173, 189), (186, 208) .

\begin{figure}[h]
\centering
\subfigure[Edges]{
\includegraphics[width=.31\textwidth]{./figure/circles_ppm_canny_32_cv}
\label{fig:hough:circle:32:cv:edge} }
\subfigure[Hough space]{
\includegraphics[width=.31\textwidth]{./figure/circles_ppm_hough_32_cv} 
\label{fig:hough:circle:32:cv:hough} } 
\subfigure[Circles found]{
\includegraphics[width=.31\textwidth]{./figure/circles_ppm_circles_32_cv}
\label{fig:hough:circle:32:cv:circles} }
\caption{Detecting circles (radius=32) on \emph{circles.ppm}.}
\label{fig:hough:circle:32:cv}
\end{figure}

\subsection{Apply weighted revote}

The local maximum finding is not easy to support multiple radii (when a set of radii is given).
Weighted revoting is tried here, which could be a feasible method in this case.
Because the weights usually converge quickly.
Instead of checking weight convergence, a fixed iteration is run for the revote process.

\begin{figure}[h]
\centering
\subfigure[Hough space]{
\includegraphics[width=.31\textwidth]{./figure/simple_circle_hough_img}
\label{fig:hough:simple_circle:32:weighted_vote:hough} }
\subfigure[Hough space after weighted revoting]{
\includegraphics[width=.31\textwidth]{./figure/simple_circle_hough_img_sup} 
\label{fig:hough:simple_circle:32:weighted_vote:hough_sup} } 
\subfigure[Circles found]{
\includegraphics[width=.31\textwidth]{./figure/simple_circle_pgm_circles_32_revote}
\label{fig:hough:simple_circle:32:weighted_vote:circles} }
\caption{Detecting circles (radius=32) on \emph{circles.ppm}.}
\label{fig:hough:simple_circle:32:weighted_vote}
\end{figure}

Figure \ref{fig:hough:simple_circle:32:weighted_vote} gives the application on a simple case. 
As shown in Figure \ref{fig:hough:simple_circle:32:weighted_vote:hough_sup}, the most likely parameter has been strengthened and the other parameters have been weakened by this revote process.
A simple threshold can be applied to find the parameter in this case.

\begin{figure}[h]
\centering
\subfigure[Hough space]{
\includegraphics[width=.31\textwidth]{./figure/circles_hough_img}
\label{fig:hough:circle:32:weighted_vote:hough} }
\subfigure[Hough space after weighted revoting]{
\includegraphics[width=.31\textwidth]{./figure/circles_hough_img_sup} 
\label{fig:hough:circle:32:weighted_vote:hough_sup} } 
\subfigure[Circles found]{
\includegraphics[width=.31\textwidth]{./figure/circles_ppm_circles_32_revote}
\label{fig:hough:circle:32:weighted_vote:circles} }
\caption{Detecting circles (radius=32) on \emph{circles.ppm}.}
\label{fig:hough:circle:32:weighted_vote}
\end{figure}

However, the result on \emph{circles.ppm} is not good as expected.
The reason might be from the code bugs.
Another guess is that the performance of this method also relies on the distribution of the edge points.

\subsection{Discussion}



\end{document}