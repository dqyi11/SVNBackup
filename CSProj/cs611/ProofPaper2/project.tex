\documentclass[12pt]{article}

% packages

%\usepackage{times} % alt: cmbright
\usepackage[top=1in, bottom=1in, left=1in, right=1in]{geometry}
\usepackage{natbib}
\usepackage{amsmath}
\usepackage{amssymb}
\usepackage{latexsym}
\usepackage{sectsty}
\usepackage{amsfonts}
\usepackage{epsfig}
\usepackage{url}
\usepackage{microtype}
\usepackage{fixmath}
\usepackage{hyperref}
\usepackage{amsthm}
\usepackage{subfigure}
\usepackage{float}
\usepackage{hyperref}

\newtheorem{lem}{Lemma}
\newtheorem{defn}{Assumption}
\newtheorem{propty}{Property}
\newtheorem{thm}{Theorem}

% references

\newcommand{\mysec}[1]{Section~\ref{sec:#1}}
\newcommand{\myapp}[1]{Appendix~\ref{app:#1}}
\newcommand{\myeq}[1]{Equation~\ref{eq:#1}}
\newcommand{\myeqp}[1]{Eq.~\ref{eq:#1}}
\newcommand{\mychap}[1]{Chapter~\ref{chap:#1}}
\newcommand{\myfig}[1]{Figure~\ref{fig:#1}}

% math conveniences

\newcommand{\g}{\,\vert\,}
\newcommand{\E}{\textrm{E}}
\newcommand{\vct}[1]{\textbf{#1}}
\newcommand{\realline}{\mathbb{R}}
\newcommand{\indpt}{\protect\mathpalette{\protect\independenT}{\perp}}
\def\independenT#1#2{\mathrel{\rlap{$#1#2$}\mkern2mu{#1#2}}}
\newcommand{\h}[1]{\textrm{H}\left( #1 \right)}
\newcommand{\half}{\frac{1}{2}}
\newcommand{\new}{\textrm{new}}

\newcommand{\mult}{\textrm{Mult}}
\newcommand{\dir}{\textrm{Dir}}
\newcommand{\discrete}{\textrm{Discrete}}
\newcommand{\Bern}{\textrm{Bern}}
\newcommand{\DP}{\textrm{DP}}
\newcommand{\GP}{\textrm{GP}}
\newcommand{\Bet}{\textrm{Beta}}

% paragraph spacing

\setlength{\parindent}{0pt}
\setlength{\parskip}{2ex plus 0.5ex minus 0.2ex}

\allsectionsfont{\sffamily\mdseries}
\paragraphfont{\sffamily\bfseries}

\usepackage{algorithm}
\usepackage{algorithmic}
\renewcommand{\algorithmicrequire}{\textbf{Input:}}
\renewcommand{\algorithmicensure}{\textbf{Output:}}


\begin{document}

\title{\textsf{Convergence of particle swarms to Pareto front in Multi-Objective Optimization}}
\author{\textsf{Daqing Yi}}
\date{\textsf{CS611 Assignment 2}}

\maketitle

%\section{Requirement}
%\begin{itemize}
%\item Write a concise, formal and accurate statement of your theorem 
%\item Include with this a set of brief definitions (formal, if possible, informal when necessary for brevity) that will allow a reader to understand the content and significance of your statement
%\end{itemize}


\section{Definitions}

\begin{prop}[\textbf{Minimum Problem}]
All the optimization functions in this paper are all defined for minimum problems.
\end{prop}
A solver of a minimum problem can be converted to a solver of a maximum problem easily.

\begin{mydef}[\textbf{Multi-Objective Optimization}]
\label{def:multi_opt}
Find the vector $ \vec{x}^{*} = \left[ {x}_{1}^{*}, {x}_{2}^{*}, \cdots {x}_{n}^{*} \right]^{T} \in \Omega $ which will satisfy the $ m $ inequality constraints
\begin{equation}
\label{eq:mo_ineq_constraint}
g_{i}(\vec{x}) \geq 0, i = 1, 2, \cdots , m, 
\end{equation}
the $ p $ equality constraints
\begin{equation}
\label{eq:mo_eq_constraint}
h_{i}(\vec{x}) \geq 0, i = 1, 2, \cdots , p, 
\end{equation}
and will optimize the vector function
\begin{equation}
\label{eq:mo_obj}
\min_{\Omega} \vec{f}(\vec{x}) = \min_{\Omega} \left[ f_{1}(\vec{x}), f_{2}(\vec{x}) \cdots f_{k}(\vec{x}) \right]^{T},
\end{equation}
where $ \vec{x} = \left[ x_{1}, x_{2}, \cdots x_{n} \right]^{T} $ is the vector of decision variables.
\end{mydef}

\begin{mydef}[\textbf{Pareto Optimal}]
\label{def:pareto_opt}
$ \vec{x}^{*} \in \Sigma $ is \emph{Pareto optimal} if for every $ \vec{x} \in \Omega $ and $ I = \{ 1, 2, \cdots k \} $, either 
\begin{equation}
\label{eq:po_eq}
\forall i \in I, f_{i}(\vec{x}) = f_{i}(\vec{x}^{*})
\end{equation}
or there is at least one $ i \in I $ such that
\begin{equation}
\label{eq:po_g}
f_{i}(\vec{x}) > f_{i}(\vec{x}^{*})
\end{equation}
for a minimum problem.
\end{mydef}


\begin{mydef}[\textbf{Pareto optimal set}]
\label{def:pareto_opt_set}
For a given Multi-Objective Optimization problem $ \vec{f}(\vec{x}) $, the \emph{Pareto optimal set} $ P^{*} $ is defined as
\begin{equation}
\label{eq:pa_opt_set}
P^{*} := \{ \vec{x} \in \Omega \mid \neg \exists \vec{x}' \in \Omega (\vec{f}(\vec{x}') \preceq  \vec{f}(\vec{x})) \},
\end{equation}
in which $ \vec{f}(\vec{x}') \preceq \vec{f}(\vec{x})) $ means $ \forall i \in I, f_{i}(\vec{x}') \leq f_{i}(\vec{x}) $.

\end{mydef}

\begin{mydef}[\textbf{Pareto Front}]
\label{def:pareto_front}
For a given Multi-Objective Optimization problem $ \vec{f}(\vec{x}) $ and Pareto optimal set $ P^{*} $, the \emph{Pareto front} $ PF^{*} $ is defined as
\begin{equation}
\label{eq:pa_front}
PF^{*} := \{ \vec{u} = \vec{f}(\vec{x}) \mid \vec{x} \in P^{*}  \}.
\end{equation}
\end{mydef}

\begin{mydef}[\textbf{Particle Swarm Optimization}]
\label{def:pso}
A \emph{particle} is a data structure that maintains current position $ \hat{\vec{x}} $ and current velocity  $ \hat{\vec{v}} $. Denote $ p_{best} $ for ``personal best'', which is the most fit position remembered of a particle. Denote $ g_{best} $ for ``global best'', which is most fit $ p_{best} $ among all particles. 
\begin{enumerate}
\item Scattering particles in decision space stochastically
\item Each particle update its location over time \\
\begin{equation}
\label{eq:up_vel}
\vec{v}_{t+1} = \vec{v}_{t} + \phi_{1} U() (p_{best} - \vec{x}_{t}) + \phi_{2} U() (g_{best} - \vec{x}_{t}),
\end{equation}
\begin{equation}
\label{eq:up_pos}
\vec{x}_{t+1} = \vec{x}_{t} + \vec{v}_{t+1},
\end{equation}
in which $ U() $ is sampling from uniform distribution.
\end{enumerate}
\end{mydef}

\section{Hypothesis}

\begin{hyp}[Convergence]
\label{thm:convergence}
Applying \emph{particle swarm optimization} \textsuperscript{\ref{def:pso}} algorithm to a \emph{multi-objective optimization} \textsuperscript{\ref{def:multi_opt}} problem, all the particles will converge to \emph{Pareto front} \textsuperscript{\ref{def:pareto_front}} of the problem after enough run.
\end{hyp}

Note: Intuitively, the convergence should depend on parameter selection and the way of measuring fitness for each particles, which will be stated clearly in rough draft of the proof.

%\bibliographystyle{apalike}
%\bibliography{reference}

\end{document}
